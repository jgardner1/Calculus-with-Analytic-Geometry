\section{Operations with Functions.}\label{sec 1.3}
If $f$ and $g$ are two functions,
a new function $f(g)$,
called the \dt{composition} of $g$ with $f$,
is defined by
\[
(f(g))(x) = f(g(x)).
\]
For example,
if $f(x) = x^3 - 1$ and $g(x) = \frac{x + 1}{x - 1}$,
then
\begin{eqnarray}
\label{eq1.3.1}
(f(g))(x)
&=& f(g(x))
 = (g(x))^3 - 1 \\
&=& \biggl( \frac{x + 1}{x - 1}\biggr)^3 - 1
 = \frac{2(3x^2 + 1)}{(x - 1)^3} . 
\end{eqnarray}
The composition of two functions
is the function obtained by applying one after the other.
If $f$ and $g$ are regarded as computing machines,
then $f(g)$ is the composite machine
constructed by feeding the output of $g$
into the input of $f$
as indicated in Figure \figref{1.17}.
\putfig{3truein}{scanfig1_17}{}{fig 1.17}

In general it is not true that $f(g) = g(f)$.
In the above example we have
\begin{eqnarray}
\label{eq1.3.2}
(g(f))(x)
&=& g(f(x)) 
 = \frac{f(x) + 1}{f(x) - 1} \\
&=& \frac{(x^3 - 1) + 1}{(x^3 - 1) - 1}
 = \frac{x^3}{x^3 - 2} ,  
\end{eqnarray}
and the two functions are certainly not the same.
In terms of ordered pairs the composition $f(g)$ of $g$ with $f$
is formally defined to be
the set of all ordered pairs $(a, c)$
for which there is an element $b$
such that $b = g(a)$ and $c = f(b)$.

If $f$ and $g$ are two real-valued functions,
we can perform the usual arithmetic operations of
addition, subtraction, multiplication, and division.
Thus for the functions
$f(x) = x^3 - 1$ and $g(x) = \frac{x + 1}{x - 1}$,
we have
\begin{eqnarray*}
f(x) + g(x)
&=& x^3 - 1 + \frac{x + 1}{x - 1} ,     \\
f(x) - g(x)
&=& x^3 - 1 - \frac{x + 1}{x - 1} ,     \\
f(x)g(x)
&=& (x^3 -1) \frac{x + 1}{x - 1} ,      \\
&=& (x^2 + x + 1)(x + 1)   \provx{if $x \neq 1$},   \\
f(x)/g(x)
&=& \frac{x^3 - 1}{\frac{x + 1}{x - 1}} \\
&=& \frac{(x^3 - 1)(x - 1)}{x + 1}.
\end{eqnarray*}
Just as with the composition of two functions,
each arithmetic operation
provides a method of constructing a new function
from the two given functions $f$ and $g$.
The natural notations for these new functions are
$f + g$, $f - g$, $fg$, and $\frac{f}{g}$.
They are defined by the formulas 
\begin{eqnarray*}
(f + g)(x)
&=& f(x) + g(x), \\
(f - g)(x)
&=& f(x) - g(x),  \\
(fg)(x)
&=& f(x)g(x),     \\
{\frac{f}{g}}(x)
&=& \frac{f(x)}{g(x)} \provx{if $g(x) \neq 0$}.
\end{eqnarray*}
The product function $fg$ should not be confused with
the composite function $f(g)$.
For example, if $f(x) = x^5$ and $g(x) = x^3$,
then we have $(fg)(x) = f(x)g(x) = {x^5} \cdot {x^3} = x^8$,
whereas
\[
(f(g))(x) = f(g(x)) = (x^3)^5 = x^{15}.
\]
We may also form the product $af$ of an arbitrary real number $a$
and real-valued function $f$.
The product function is defined by
\[
(af)(x)= af(x).
\]


\begin{example}\label{exam 1.3.1}
Let functions $f$ and $g$ be defined by
$f(x) = x - 2$ and $g(x) = x^2 - 5x + 6$.
Draw the graphs of $f$, $g$, $2f$, and $f + g$.
We compute the function values
corresponding to several different numbers $x$ in
Tables \tabref{1.3} and \tabref{1.4}.
The resulting graphs of $f$ and $g$ are, respectively,
the straight line and parabola shown in Figure \figref{1.18}(a).
\putfig{4.5truein}{scanfig1_18}{}{fig 1.18}
It turns out that the graphs of $2f$ and $f + g$
are also a straight line and a parabola.
They are drawn in Figure \figref{1.18}(b).
To see why the graph of $f + g$ is a parabola,
observe that 
\begin{eqnarray*}
(f + g)(x)
&=& f(x) + g(x)
 = (x - 2) + (x^2 - 5x + 6) = x^2 - 4x + 4 \\
&=& (x - 2)^2.
\end{eqnarray*}
It follows that $f + g$ is very much like
the function defined by $y = x^2$.
Instead of simply squaring a number,
$f + g$ first subtracts $2$ and then squares.
Its graph will be just like that of $y = x^2$
except that it will be shifted two units to the right.
\end{example}

\begin{table}
\[
\begin{array}{r|r|c}
\hline
  x   &  f(x)  &  2f(x)  \\
\hline
  0   &  -2    &  -4     \\
  1   &  -1    &  -2     \\
  2   &   0    &   0     \\
  3   &   1    &   2     \\
\hline
\end{array}
\]
\caption{}
\label{table 1.3}
\end{table}

\begin{table}
\centering
\[
\begin{array}{r|c}
\hline 
x               &  g(x)     \\
\hline 
0               &  6        \\ 
5               &  6        \\
\frac{5}{2}     &  -\frac{1}{4}   \\
1               &  2        \\
4               &  2        \\
\hline
\end{array}
\]
\caption{}
\label{table 1.4}
\end{table}

Up to this point
we have used the letters
$f$, $g$, $h$, $F$, $G$, and $H$
to denote functions,
and the letters $x$, $y$, $a$, $b$, and $c$
to denote elements of sets---usually real numbers.
However, the letters in the second set
are sometimes also used as functions.
This occurs, for example,
when we speak of $x$ as a real variable.
As such, it not only is the name of a real number
but also can take on many different values:
$5$, or $-7$, or $\pi$, or \ldots.
Thus the variable $x$ is a function.
Specifically, it is the very simple function
that assigns the value $5$ to the number $5$,
the value $-7$ to the number $-7$,
the value $\pi$ to $\pi$, \ldots.
For every real number $a$, we have
\[
x(a) = a.
\]
This function is called the \dt{identity function}.

Suppose, for example,
that $s$ is used to denote the distance
that a stone falling freely in space has fallen.
The value of $s$ increases as the stone falls
and depends on the length of time $t$ that it has fallen
according to the equation
$s= {\frac{1}{2}}g{t^2}$,
where $g$ is the constant gravitational acceleration.
(This formula assumes no air resistance,
that the stone was at rest at time $t = 0$,
and that distance is measured from the starting point.)
Thus $s$ has the value ${\frac{9}{2}}g$
if $t$ has the value $3$,
and, more generally,
the value ${\frac{1}{2}}g{a^2}$
when $t$ has the value $a$.
If we consider $t$ to be another name for the identity function,
then $s$ may be regarded as the function whose value is
\[
s(a) = {\frac{1}{2}}{g{a^2}} = {\frac{1}{2}}{g(t(a))^2}
\]
for every real number $a$.
The original equation $s = {\frac{1}{2}}g{t^2}$ then states
the relation between the two functions $s$ and $t$.
The fact that $s$ and $t$ take on different values
is also expressed by referring to them as variables.
A \dt{variable} is simply a name of a function.
In our example $s$ is called a dependent variable,
and $t$ an independent variable,
because the values of $s$ depend on those of $t$
according to $s = {\frac{1}{2}}g{t^2}$.
Thus an \dt{independent variable}
is a name for the identity function,
and a \dt{dependent variable} is one that is not independent.

A real variable is therefore a name of a real-valued function.
Since the arithmetic operations of
addition, subtraction, multiplication, and division
have been defined for real-valued functions,
they are automatically defined for real variables.

We shall generally use the letter $x$
to denote an independent variable.
This raises the question:
How does one tell whether an occurrence of $x$
denotes a real number or the identity function?
The answer is that the notation alone does not tell,
but the context and the reader's understanding should.
However, a more practical reply
is that it doesn't really make much difference.
We may regard $f(x)$ as either
the value of the function $f$ at the number $x$
or as the composition of $f$ with the variable $x$.
If $x$ is an independent variable,
the function  $f(x)$ is then the same thing as $f$.

\begin{example}\label{exam 1.3.2}
The conventions that we have adopted
concerning the use of variables
give our notations a flexibility
that is both consistent and extremely useful.
Consider, for example, the equation
\[
y= 2x^2 - 3x.
\]
On the one hand,
we may consider the subset of $\R^2$, pictured in
Fifure \figref{1.19},
\putfig{3.5truein}{scanfig1_19}{}{fig 1.19}
that consists of all ordered pairs $(x, y)$ such that $y = 2x^2 - 3x$.
This subset is a function $f$
whose value at an arbitrary real number $x$
is the real number $f(x) = 2x^2 - 3x$.
Alternatively,
we may regard $x$ as an independent variable,
i.e., the identity function.
The composition of $f$ with $x$ is then the
function $f(x) = 2x^2 - 3x$, whose value at $2$,
for instance, is
\[
(f(x))(2) = f(x(2)) = f(2) = 8 - 6 = 2.
\]
A third interpretation
is that $y$ is a dependent variable that depends on $x$
according to the equation $y = 2x^2 - 3x$.
That is, $y$ is the name of the function $2x^2 - 3x$.
\end{example}

\begin{example}\label{exam 1.3.3}
Let $F$ be the function defined by
$F(x) = x^3 + x + 1$. If $u = \sqrt{x - 2}$,
then
\begin{eqnarray*}
F(u)
&=&  u^3 + u + 1  \\
&=&  (x - 2)^{3/2} + (x - 2)^{1/2} + 1.
\end{eqnarray*}
If we denote the function $F(x)$ by $w$, then 
\[
u + w =  \sqrt{x - 2} + x^3 + x + 1,
\]
\[
uw = (x - 2)^{1/2} (x^3 + x + 1).
\]
On the other hand,
we may let $G$ be the function defined by
$G(x) = \sqrt{x - 2}$
for every real number $x \geq 2$.
Then $G + F$ and $GF$ are the functions defined, respectively, by
\begin{eqnarray*}
(G + F)(x)
&=&  G(x) + F(x) \\
&=&  \sqrt{x - 2} + x^3 + x + 1, \\
(GF)(x)
&=& G(x)F(x) \\
&=&  (x - 2)^{1/2} (x^3 + x + 1).
\end{eqnarray*}
\end{example}

To say that $a$ is a real \dt{constant}
means first that it is a real number.
Second, it may or may not matter which real number $a$ is,
but it is fixed for the duration of the discussion in which it occurs.
Similarly, a \dt{constant function} is one which takes on just one value;
i.e., its range consists of a single element.
For example, consider the constant function $f$ defined by
\[
f(x) = 5, \;\;\; - \infty < x < \infty.
\]
The graph of $f$ is the straight line parallel to the $x$-axis
that intersects the $y$-axis in the point (0, 5);
see Figure \figref{1.20}.
\putfig{3truein}{scanfig1_20}{}{fig 1.20}
We shall commonly use lower-case letters
at the beginning of the alphabet,
e.g., $a$, $b$, $c$,...,
to denote both constants and constant functions.

\begin{example}\label{exam 1.3.4}
Consider the function $ax + b$,
where $a$ and $b$ are constants,
$a \neq 0$,
and $x$ is an independent variable.
The graph of this function is a straight line
that cuts the $y$-axis at $b$ and the $x$-axis at $-\frac{b}{a}$.
It is drawn in Figure \figref{1.21}.
\putfig{2truein}{scanfig1_21}{}{fig 1.21}
This function is the sum of the constant function $b$
and the function which is the product of the constant function $a$
and the identity function $x$.
\end{example}
