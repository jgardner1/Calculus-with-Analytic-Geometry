\chapter{Functions, Limits, and Derivatives} \label{chp 1}

\section{Real Numbers, Inequalities, Absolute Values.}\label{sec 1.1} 

Calculus deals with numerical-valued quantities and,
in the beginning, with quantities whose values are real numbers.
Some understanding of the basic set $\R$ of all real numbers
is therefore essential.

A \dt{real number} is one that can be written as a decimal:
positive or negative or zero, terminating or nonterminating.
Examples are
\[
1, \;-5, \;0, \;14,
\]
\[
\frac{2}{3} = 0.666666 \ldots, \; \frac{3}{8} = 0.375,
\]
\[
\sqrt{2} = 1.4142 \ldots,
\]
\[
-\pi = -3.141592\ldots,
\]
\[
176355.14233333 \ldots
.
\]

The most familiar subset of $R$ is the set $\Z$ of \dt{integers}.
These are the numbers
\begin{equation}
\ldots, -3, -2, -1, 0, 1, 2, 3,\ldots .   
\label{eq1.1.1}
\end{equation}
Another subset is the set $\Q$ of all rational numbers.
A real number $r$ is \dt {rational}
if it can be expressed as the ratio of two integers,
more precisely,
if $r = \frac{m}{n}$,
where $m$ and $n$ are integers and $n \neq 0$.
Since every integer $m$ can be  written $\frac{m}{1}$,
it follows that every integer is also a rational number.
A scheme, analogous to \eqref{1.1.1},
which lists all the positive rational
numbers is the following:

\begin{equation}
\label{eq1.1.2}
\begin{array}{ccccc}
\frac{1}{1}, & \frac{2}{1}, & \frac{3}{1}, & \frac{4}{1}, & \ldots\\
\frac{1}{2}, & \frac{2}{2}, & \frac{3}{2}, & \frac{4}{2}, & \ldots\\
\frac{1}{3}, & \frac{2}{3}, & \frac{3}{3}, & \frac{4}{3}, & \ldots\\
\vdots
\end{array}
\end{equation} 
Of course there are infinitely many repetitions in this presentation since,
for example,
$\frac{2}{1} = \frac{4}{2} = \frac{6}{3} = \ldots.$
An unsophisticated guess would be that all real numbers are rational.
There are, however, many famous proofs that this is not so.
For example,
a very simple and beautiful argument shows that $\sqrt 2$ is not rational.
(See Problem \exref{1.1.13} at the end of this section.)
It is not hard to prove that a real number is rational
if and only if its decimal expansion beyond some digit
consists of a finite sequence of digits repeated forever.
Thus the numbers
\[
1.71349213213213213213 \ldots \mbox{(forever)} ,
\]
\[
1.500000000 \ldots \mbox{(forever)}
\]
are rational, but
\[
0.101001000100001000001 \ldots \mbox{(etc.)}
\]
is not.

The fundamental algebraic operations on real numbers
are addition and multiplication:
For any two elements $a$ and $b$ in $\R$,
two elements $a + b$ and $ab$ in $\R$ are uniquely determined.
These elements,
called the \dt{sum} and \dt{product} of $a$ and $b$, respectively,
are defined so that the following six facts are true:


\begin{axiom}[Associative Laws]
\label{axiom.i}
\[
a + (b + c) = (a + b) + c,
\]
\[
a(bc) = (ab)c.
\]
\end{axiom}

\begin{axiom}[Commutative Laws]
\label{axiom.ii}
\[
a + b = b + a,
\]
\[
ab= ba.
\]
\end{axiom}

\begin{axiom}[Distributive Law]
\label{axiom.iii}
\[
(a + b) c = ac + bc.
\]
\end{axiom}

\begin{axiom}[Existence of Identities]
\label{axiom.iv}
$\R$ contains two distinct elements $0$ and $1$
with the properties that $0 + a = a$ and $1 \cdot a = a$
for every $a$ in $\R$.
\end{axiom}


\begin{axiom}[Existence of Subtraction]
\label{axiom.v}
For every $a$ in $\R$,
there is an element in $\R$ denoted by $-a$
such that $a + (-a) = 0$.
\begin{note}
$a - b$ is an abbreviation of $a + (-b)$.
\end{note}
\end{axiom}


\begin{axiom}[Existence of Division]
\label{axiom.vi}
For every $a \neq 0$ in $\R$,
there is an element in $\R$ denoted by $a^{-1}$ or $\frac{1}{a}$
such that $aa^{-1} = 1$.
\begin{note}
$\frac{a}{b}$ is an abbreviation of $ab^{-1}$.
\end{note}
\end{axiom}

Addition and multiplication are here introduced as binary operations.
However, as a result of the associative law of addition,
$a + b + c$ is defined to be the
common value of $(a + b) + c$ and $a + (b + c)$.
In a like manner we may define the triple product $abc$ and,
more generally,
$a_{1} + \ldots + a_{n}$ and $a_{1} \ldots a_{n}$.
Many theorems of algebra are consequences of the above six facts,
and we shall assume them without proof.
They are, in fact,
frequently taken as part of a set of axioms for $\R$.

Another essential property of the real numbers is that of order.
We write $a < b$
as an abbreviation of the statement that $a$ is less than $b$.
Presumably the reader, given two decimals,
knows how to tell which one is the smaller.
The following four facts simply recall
the basic properties governing inequalities.
On the other hand,
they may also be taken as axioms for an abstractly defined relation
between elements of $\R$,
which we choose to denote by $<$.
 
\begin{axiom}[Transitive Law]
\label{axiom.vii}
If $a < b$ and $b < c$, then $a < c$.
\end{axiom}

\begin{axiom}[Law of Trichotomy]
\label{axiom.viii}
For every real number $a$,
one and only one of the following alternatives holds:
$a = 0$, or $a < 0$, or $0 < a$.
\end{axiom}

\begin{axiom}
\label{axiom.ix}
If $a < b$, then $a + c < b + c$.
\end{axiom}

\begin{axiom}
\label{axiom.x}
\emph{If $a < b$ and $0 < c$, then $ac < bc$.}
\end{axiom}
 
Note that each of the above Axioms except \ref{axiom.vi}
remains true when restricted to the set $\Z$ of integers.
Moreover,
all the axioms are true for the set $\Q$ of rational numbers.
Hence as a set of axioms for $\R$,
they fail to distinguish between two very different sets:
$\R$ and its subset $\Q$.
Later in this section we shall add one more item to the list,
which will complete the algebraic description of $\R$.

A real number $a$ is if \dt{positive} $0 < a$
and \dt{negative} if $a < 0$.
Since the relation ``greater than" is just as useful as ``less than,"
we adopt a symbol for it, too,
and abbreviate the statement that $a$ is greater than $b$
by writing $a > b$.
Clearly $a > b$ if and only if $b < a$.
Axiom \ref{axiom.x},
when translated into English,
says that the direction of an inequality is preserved
if both sides are multiplied by the same positive number.
Just the opposite happens if the number is negative:
The inequality is reversed.
That is,

\begin{prop}
\label{thm 1.1.1}
If $a < b$  and  $c < 0$, then $ac > bc$.
\end{prop}
\begin{proof}
Since $c < 0$, Axioms
\ref{axiom.iv},
\ref{axiom.v},
and
\ref{axiom.ix}
imply
\[
0 = c + (-c) < 0 + (- c) = - c
.
\]
So $-c$ is positive.
Hence by $(x)$, we get $-ac < -bc$.
By Axiom
\ref{axiom.ix}
again, 
\[
-ac + (bc + ac) < -bc + (bc + ac).
\]
Hence $bc < ac$,
and this is equivalent to $ac > bc$.
\end{proof}

Two more abbreviations complete the mathematician's array of symbols
for writing inequalities:

$a \leq b$ means $a < b$ or $a=b$,

$a \geq b$ means $a > b$ or $a=b$.

The geometric interpretation of the set $\R$ of all real numbers
as a straight line is familiar to anyone who has ever used a ruler,
and it is essential to an understanding of calculus.
To describe the assignment of points to numbers,
consider an arbitrary straight line $L$,
and choose on it two distinct points,
one of which we assign to, or identify with, the number $0$,
and the other to the number $1$.
(See Figure \figref{1.1}.)
\putfig{2truein}{scanfig1_1}
{A line $L$ with two distinguished point $0$ and $1$.}{fig 1.1}
The rest is automatic.
The scale on $L$ is chosen so that the unit of distance is
the length of the line segment between the points $0$ and $1$.
Every positive number $a$ is assigned the point on the side of $0$
containing $1$ which is $a$ units of distance from $0$.
Every negative number $a$ is assigned the point on the side of $0$
not containing $1$ which is $-a$ units of distance from $0$.
Note that if $L$ is oriented so that $1$ lies to the right of $0$,
then
\emph{
for any two numbers $a$ and $b$ (positive, negative, or zero),
$a < b$ if and only if $a$ lies to the  left of $b$.}
A line which has been identified with $\R$ under a correspondence
such as the one just described is called a \dt{real number line}.
(See Figure \figref{1.2}.)

\putfig{3truein}{scanfig1_2}
{A real number line.}{fig 1.2}

An \dt{interval}
is a subset $I$ of $\R$ with the property that
whenever $a$ and $c$ belong to $I$ and $a \leq b \leq c$,
then $b$ also belongs to $I$.
Geometrically an interval is a connected piece of a real number line.
A number d is called a \dt{lower bound} of a set $S$ of real numbers
if $d \leq s$ for every $s$ in $S$.
It is an \dt{upper bound} of $S$ if $s \leq d$ for every $s$ in $S$.
A given subset of $\R$, and in particular an interval,
is called \dt{bounded} if it has both an upper and lower bound.
There are four different kinds of bounded intervals:
\begin{quote}
$(a, b)$, the set of all numbers $x$ such that  $a < x < b$;

$[a, b]$, the set of all numbers $x$ such that  $a \leq x \leq b$;

$[a, b)$, the set of all numbers $x$ such that  $a \leq x < b$;

$(a, b]$, the set of all numbers $x$ such that  $a < x \leq b$.
\end{quote}
In each case the numbers $a$ and $b$
are called the \dt{endpoints} of the interval.
The set $[a, b]$ contains both its endpoints,
whereas $(a, b)$ contains neither one.
Clearly $[a, b)$ contains its left endpoint but not its right one,
and an analogous remark holds for $(a, b]$.

It is important to realize that
there is no element $\infty$ (infinity) in the set $\R$.
Nevertheless, the symbols $\infty$ and $-\infty$ are commonly used
in denoting unbounded intervals.
Thus
\begin{quote}
$(a, \infty)$ is the set of all numbers $x$ such that $a < x$;

$[a, \infty)$ is the set of all numbers $x$ such that $a \leq x$;

$(-\infty, a)$ is the set of all numbers $x$ such that $x < a$;

$(-\infty, a]$ is the set of all numbers $x$ such that $x \leq a$;

$(-\infty, \infty)$ is the entire set $\R$.
\end{quote}
The symbols $\infty$ and $-\infty$ also appear frequently
in inequalities although they are really unnecessary,
because, for example,
\[
\begin{array}{r}
-\infty < x < a \;\;\; \mbox{is equivalent to $x < a$},  \vspace{.1in} \\
a \leq x < \infty \;\;\;\mbox{is equivalent to $a \leq x$}, 
\end{array}
\]
etc.
Since $\infty$ is not an element of $\R$,
we shall never use the notations $[a, \infty], x \leq \infty$, etc.
An unbounded interval has either one endpoint or none;
in each of the above cases it is the number $a$.
We call an interval \dt{open} if it contains none of its endpoints,
and \dt{closed} if it contains them all.
Thus, for example, $(a, b)$ and $(-\infty, a)$ are open,
but $[a, b]$ and $[a, \infty)$ are closed.
The intervals $[a, b)$ and $(a, b]$ are neither open nor closed,
although they are sometimes called half-open or half-closed.
Since $(-\infty, \infty)$ has no endpoints,
it vacuously both does and does not contain them.
Hence $(-\infty, \infty)$ has the dubious distinction
of being both open and closed.

\putfig{4truein}{scanfig1_3}{Types of intervals.}{fig 1.3}

\begin{example}
\label{exam 1.1.1}
Draw the intervals
$[0, 1]$, $[-1, 4)$, $(2, \infty$), $(-\infty, -1]$, $(-1, 3)$,
and identify them as open, closed, neither,
or both (see Figure \figref{1.3}).
\end{example}

It is frequently necessary to talk about the size of a real number
without regard to its sign, not caring whether it is positive or negative.
This happens often enough to warrant a definition and special notation:
The \dt{absolute value} of a real number $a$ is denoted by $|a|$
and defined by
\[
|a| = \left \{  \begin{array}{rl}
a    & \mbox{if $a \geq 0$}, \\
-a   & \mbox{if $a < 0$}.
                    \end{array}
\right. 
\]
Thus $|3| = 3$, $|0| = 0$, $|- 3| = 3$.
Obviously,
\emph{the absolute value of a real number cannot be negative.}
Geometrically,
$|a|$ is the distance between the points $0$ and $a$
on the real number line.
A generalization that is of extreme importance is the fact that $|a - b|$
\emph{is the distance between the
points $a$ and $b$ on the real number line for any two numbers
$a$ and $b$ whatsoever.}
Probably the best way to convince oneself that this is true
is to look at a few illustrations (see Figure \figref{1.4}).

\putfig{4truein}{scanfig1_4}
{Computing distances with the absolute value.}{fig 1.4}

\begin{example}
\label{exam 1.1.2}
Describe the set $I$ of all real numbers $x$
such that $|x - 5| < 3$.
For any number $x$,
the number $|x - 5|$ is the distance between $x$ and $5$
on a real number line (see Figure \figref{1.5}).
\putfig{3truein}{scanfig1_5} {An open ball in a one-dimensional space.}{fig 1.5}
That distance will be less than $3$
if and only if $x$ satisfies the inequalities $2 < x < 8$.
We conclude that $I$ is the open interval $(2, 8)$.
\end{example}


There is an alternative way of writing the definition
of the absolute value of a number $a$
which requires only one equation:
We do not have to give separate definitions for positive and negative $a$.
This definition uses a square root,
and before proceeding to it,
we call attention to the following mathematical custom:
Although every positive real number a has two square roots,
in this book
\emph{the expression $\sqrt a$ always denotes the positive root}.
Thus the two solutions of the equation $x^2 = 5$ are
$\sqrt 5$ and $-\sqrt 5$.
Note that the two equations
\[
x^2 = a
\]
and
\[
x = \sqrt a 
\]
are not equivalent.
The second implies the first, but not conversely.
On the other hand,
\[
x^2 = a
\]
and
\[
|x| = \sqrt a 
\]
are equivalent.
Having made these remarks, we observe that

\begin{prop} 
\label{thm 1.1.2}
\[
 |a| = \sqrt {a^2} .   
\]
\end{prop}

The formulation \thref{1.1.2} is a handy one
for establishing two of the basic properties of absolute value.
They are

\begin{prop} 
\label{thm 1.1.3}
\[
 |ab| = |a| |b|.
\]
\end{prop}

\begin{prop} 
\label{thm 1.1.4}
\[
 |a + b| \leq |a| + |b|.
\]
\end{prop}

\begin{proof}
Since $(ab)^2 = {a^2}{b^2}$
and since the positive square root of a product of two positive numbers
is the product of their positive square roots,
we get 
\[
|ab| = \sqrt {(ab)^2}  = \sqrt {{a^2}{b^2}}  = \sqrt{a^2} \sqrt{b^2} = |a| |b|.
\]

To prove \thref{1.1.4},
we observe, first of all, that $ab \leq |ab|$.
Hence
\[
a^2 + 2ab + b^2 \leq a^2 + 2|ab| + b^{2} = |a|^2 + 2|a| |b| + |b|^2.
\]
Thus,
\[
|a + b|^{2} = (a + b)^{2} \leq (|a| + |b|)^{2}.
\]
By taking the positive square root of each side of the inequality
(see Problem \exref{1.1.9}), we get \thref{1.1.4}.
\end{proof}

As remarked above, our list of Axioms
\ref{axiom.i} through \ref{axiom.x}
about the set $\R$ of real numbers is incomplete.
One important property of real numbers
that together with the others gives a complete characterization
is the following:


\begin{axiom}[Least Upper Bound Property]
\label{axiom.xi}
Every nonempty subset of $\R$ which has an upper bound
has a least upper bound.
\end{axiom}

Suppose $S$ is a nonempty subset of $\R$ which has an upper bound.
What Axiom \ref{axiom.xi} says
is that there is some number $b$ which
(1) is an upper bound, i.e., $s \leq b$ for every $s$ in $S$, and
(2) if $c$ is any other upper bound of $S$, then $b \leq c$.
It is hard to see at first how such a statement can be so significant.
Intuitively it says nothing more than this:
If you cannot go on forever, you have to stop somewhere.
Note, however, that the rational numbers do \emph{not} have this property.
The set of all rational numbers less than the irrational number $\sqrt2$
certainly has an upper bound.
In fact, each of the numbers
$2$, $1.5$, $1.42$, $1.415$, $1.4143$, and $1.41422$ is an upper bound.
However, for every rational upper bound,
there will always exist a smaller one.
Hence there is no rational least upper bound.
