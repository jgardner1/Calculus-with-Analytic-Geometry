\begin{exercises}

\ex{10.7.1}
In each of the following, draw the curve defined by the
equation $r = f(\theta)$ in polar coordinates.
Show the region $R$ bounded by the curve and the
lines $\theta = a$ and $\theta = b$, and compute
its area.
\begin{exenum}
\x
$r=4\cos \theta$, $a=0$ and $b=\frac\pi2$.
\x
$r=3(1+\cos\theta)$, $a=0$ and $b=\pi$.
\x
$r=3(1+\sin\theta)$, $a=0$ and $b=\frac\pi2$.
\x
$r=\frac2{\cos\theta}$, $a=-\frac\pi4$ and $b=\frac\pi4$.
\end{exenum}

\ex{10.7.2}
For each of the following equations $r=f(\theta)$
and pairs of numbers $a$ and $b$,
draw the region $R$ consisting of all points
with polar coordinates $(r,\theta)$
such that $a\leq\theta\leq b$ and
$0\leq r\leq f(\theta)$.
Compute $\mbox{\emph{area}}(R)$.
\begin{exenum}
\x
$r=4\sin\theta$, $a=0$ and $b=\pi$.
\x
$r=\frac4{\sin\theta}$, $a=\frac\pi4$ and $b=\frac{3\pi}4$.
\x
$r=2\theta$, $a=\pi$ and $b=2\pi$.
\x
$r=\frac1{2\cos\theta+3\sin\theta}$,
$a=0$ and $b=\frac\pi2$.
\x
$r=\sqrt{2\cos2\theta}$, $a=0$ and $b=\frac\pi4$.
(See Example \exampref{10.6.5}.)
\end{exenum}

\ex{10.7.3}
Identify and draw the curve defined by the equation
$r=4\sin\theta$ in polar coordinates,
and show the region $R$ bounded by the curve.
Is it true in this case that
\[
\mbox{\emph{area}}(R) = \frac12
\int_0^{2\pi} r^2 \; d\theta
?
\]
Explain your answer.

\ex{10.7.4}
Each of the following curves, defined by an equation
$r=f(\theta)$ in polar coordinates,
bounds a region $R$ in the plane.
Draw the curve and find the area of $R$.
\begin{exenum}
\x
$r=a(1+\cos\theta)$, $a>0$
\x
$r=a(1+\sin\theta)$, $a>0$
\x
$r=5$
\x
$r=2+\cos\theta$
\x
$r=4\sin\theta$
\x
$r=-4\cos\theta$.
\end{exenum}

\ex{10.7.5}
The curve defined by the equation
$r=\frac1{1+\cos\theta}$ in polar coordinates
is a parabola similar to the one discussed in 
Example \exampref{10.6.3}.
\begin{exenum}
\x
Draw the parabola, and show the region $R$
bounded by this curve and the line
$\theta=\frac\pi2$.
\x
\xlab{10.7.5b}
Express $\mbox{\emph{area}}(R)$ as a definite
integral using the integral formula for area
in polar coordinates.
\x
Evaluate the integral in part \exref{10.7.5b} using the
trigonometric substitution $z=\tan \frac\theta2$ (see
equation \eqref{7.5.1}) and the Change of Variable
Theorem for Definite Integrals.
\x
Write this curve as the graph of an equation
in $x$- and $y$-coordinates,
and thence compute \emph{area}(R).
\end{exenum}

\ex{10.7.6}
Find the area of the region which lies between the
two loops of the lima\c{c}on $r=1+2\cos \theta$.

\ex{10.7.7}
Find the area of the region bounded by the lemniscate
$r^2 = 2a^2 \cos 2\theta$.

\ex{10.7.8}
Find the area $A$ of the region which lies inside
the cardioid $r=2(1+\cos\theta)$ and outside
the circle $r=3$.

\ex{10.7.9}
The region $R$ bounded by the cardioid
$r=4(1+\sin\theta)$ is cut into two regions
$R_1$ and $R_2$ by the polar graph of the
equation $r=\frac3{\sin\theta}$.
Compute the areas of $R$, $R_1$, and $R_2$.

\ex{10.7.10}
Find the arc length of the cardioid defined by the
equation $r=a(1+\cos\theta)$,
where $a$ is an arbitrary positive constant.

\ex{10.7.11}
Consider the spiral defined in polar coordinates
by the equation $r=e^{2\theta}$.
Compute the arc length of this curve from 
$\theta=0$ to $\theta=\ln 10$.

\ex{10.7.12}
\begin{exenum}
\x
\xlab{10.7.12a}
Using the integral formula for arc length in polar
coordinates, compute the arc length of the polar
graph of the equation $r=2\sec \theta$ from
$\theta = -\frac\pi4$ to $\theta=\frac\pi4$.
\x
Identify and draw the curve in part \exref{10.7.12a},
and verify from the geometry the value obtained
for the arc length.
\end{exenum}

\ex{10.7.13}
Consider the curve defined by the equation
$r=2\cos^2 \frac\theta2$ in polar coordinates.
\begin{exenum}
\x
Find the arc length of this curve from
$\theta=0$ to $\theta=\pi$.
\x
Find the arc length of this curve from
$\theta=0$ to $\theta=2\pi$.
\end{exenum}

\end{exercises}
