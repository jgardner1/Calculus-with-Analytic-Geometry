\begin{exercises}

\ex{10.3.1}
Find the terminal point of each of the following vectors.
Draw each one as a directed line segment in
the $xy$-plane, and compute its length.
\begin{exenum}
\x
$\vec v = (-3,4)_P$, where $P = (1,0)$,
\x
$\vec u = (4, -3)_P$, where $P = (1,0)$,
\x
$\vec x = (3,0)_Q$, where $Q = (-1,-1)$,
\x
$\vec a = (4\frac12,3\frac12)_O$, where $O = (0,0)$.
\end{exenum}

\ex{10.3.2}
Let $P = (2,1)$.
Compute the terminal point of each of the following
vectors, and draw each one as an arrow in the
$xy$-plane.  The vectors $\vec u$ and $\vec v$
in parts \exref{10.3.2b}, \exref{10.3.2c},
\exref{10.3.2d}, and \exref{10.3.2e} are defined
as in part \exref{10.3.2a}.
\begin{exenum}
\x
\xlab{10.3.2a}
$\vec u = (3,-2)_P$ and $\vec v = (1,1)_P$
\x
\xlab{10.3.2b}
$\vec u + \vec v$
\x
\xlab{10.3.2c}
$\vec u - \vec v$
\x
\xlab{10.3.2d}
$3\vec v$
\x
\xlab{10.3.2e}
$\vec u + 3\vec v$.
\end{exenum}

\ex{10.3.3}
Let $P = (0,1)$, and consider the vectors
$\vec x = (2,5)_P$ and $\vec y = (1,1)_P$.
\begin{exenum}
\x
Draw the vectors $\vec x$, $\vec y$, and
$\vec x + \vec y$ in the $xy$-plane.
\x
Compute the lengths $|\vec x|$, $|\vec y|$,
and $|\vec x+\vec y|$.
\end{exenum}

\ex{10.3.4}
True or false: If $P \ne Q$, then $V_P$ and $V_Q$
are disjoint sets?

\ex{10.3.5}
Let $\vec v$ be a vector in the plane with initial
point $P$, and let $\theta$ be the angle whose
vertex is $P$, whose initial side is the vector
$(1,0)_P)$, and whose terminal side is $\vec v$.
Show that
\[
\vec v = (|v| \cos \theta, |v| \sin \theta)_P
.
\]
The angle $\theta$ is called the \dt{direction}
of the vector $\vec v$.

\ex{10.3.6}
In physics, the force acting on a particle located
at a point $P$ in the plane is represented by a vector.
The length of the vector is the magnitude of the force
(e.g., the number of pounds), and the direction
of the vector is the direction of the force
(see Problem \exref{10.3.5}).
\begin{exenum}
\x
\xlab{10.3.6a}
Draw the vector representing a force of $5$ pounds
acting on a particle at the point $(3,2)$ in a
direction of $\frac{\pi}6$ radians.
\x
What are the coordinates of the force vector
in \exref{10.3.6a}?
\end{exenum}

\ex{10.3.7}
If a particle located at a point $P$ is simultaneously
acted on by two forces $\vec u$ and $\vec v$, then the
resultant force is the vector sum $\vec u + \vec v$.
The fact that vectors are added geometrically
by constructing a parallelogram implies a corresponding
Parallelogram Law of Forces.

Suppose that a particle at the point $(1,1)$ is
simultaneously acted on by a force $\vec v$
of $10$ pounds in the direction of $\frac\pi6$
radians and a force $\vec u$ of $\sqrt{32}$
pounds in the direction of $-\frac\pi4$ radians.
\begin{exenum}
\x
Draw the parallelogram of forces, and show
the resultant force.
\x
What are the coordinates of the resultant
force on the particle?
\end{exenum}

\ex{10.3.8}
Addition and scalar multiplication are defined in the
set $\R^2$ of all ordered pairs of real numbers
by the equations
\[
(a,b) + (c,d) = (a+c,b+d)
,
\]
\[
c(a,b) = (ca,cb)
.
\]
Show that $\R^2$ is a vector space with respect
to these operations.  This fact shows that the elements
of a vector space need not necessarily be interpreted
as arrows.  The principal interpretation of $\R^2$
is that of the set of points of the plane.

\ex{10.3.9}
True or false?
\begin{exenum}
\x
The set $\R$ of all real numbers is a vector space
with respect to ordinary addition and multiplication.
\x
The set $\C$ of all complex numbers is a vector space
with respect to addition and multiplication
by real numbers.
\x
The set $V$ of all vectors in the plane is a vector
space with respect to vector addition and scalar
multiplication as defined in this section.
\end{exenum}

\end{exercises}
