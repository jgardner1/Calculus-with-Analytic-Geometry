\begin{exercises}

\ex{5.3.1}
Prove that if $f$ is a strictly increasing function,
then the inverse function $f^{-1}$
is also strictly increasing.

\ex{5.3.2}
Does the assertion in Problem \exref{5.3.1}
remain true if ``increasing'' is replaced by
``decreasing''?

\ex{5.3.3}
Show by giving an example that a strictly
increasing function is not necessarily continuous.

\ex{5.3.4}
Give an example of a differentiable strictly
increasing function defined for all values of $x$
whose inverse is not differentiable everywhere.

\ex{5.3.5}
Show that Theorem \thref{5.3.4} is geometrically
obvious.
[\emph{Hint:} The derivative is the slope of the tangent
line, and the graph of $y = f(x)$ is the same as
that of $x = f^{-1}(y)$.]

\ex{5.3.6}
Supply the details which prove that Theorem \thref{5.3.2}
is equivalent to \thref{5.3.1} [i.e., to the conjunction
of \thref{5.3.1} and its companion].

\end{exercises}
