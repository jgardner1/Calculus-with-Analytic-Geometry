\begin{exercises}

\ex{11.3.1}
Find the general solution of each of the following
differential equations.
\begin{exenum}
\x
$\deriv2y + \dydx - 2y = 5{e^{-x}}$
\x
$(D+2)(D-1)y=6e^{-2x}$
\x
$(D^2-3D+2)y = 4x+3$
\x
$\deriv2y + y = e^x$
\x
$(D^2+1)y = x^2+1$.
\end{exenum}

\ex{11.3.2}
Using equations \eqref{11.3.1} and \eqref{11.3.2},
prove that, if $L$ is a linear operator, then
\[
L(y_1 - y_2) = L(y_1) - L(y_2)
.
\]

\ex{11.3.3}
Show that equations \eqref{11.3.1} and \eqref{11.3.2}
can be replaced by a single equation.
That is, prove that a function $L$ is a linear
operator if and only if
\[
L(ay_1+by_2) = aLy_1 + bLy_2
.
\]

\ex{11.3.4}
Prove \thref{11.3.2}; i.e., if $L_1$ and $L_2$
are linear operators, then the composition $L_1L_2$
is also a linear operator.

\ex{11.3.5}
Prove the second equation in \thref{11.3.3}, i.e.,
the distributive law
$(L_1+L_2)L_3 = L_1L_3+L_2L_3$.

\ex{11.3.6}
It might at first seem more natural to define the product
of two linear operators $L_1$ and $L_2$ by the equation
\[
(L_1L_2)y = (L_1y)(L_2y)
.
\]
(This \emph{is} the way the product of two
real-valued functions is defined.)
Using this definition, show that, if $D$ is the derivative,
the $D^2$ is not a linear operator.

\ex{11.3.7}
Let $f(x)$ be a given function and $L$ a linear operator.
Define $f(x)L$ by the equation
\[
(f(x)L)y =f(x)(L_y)
.
\]
Show that $f(x)L$ satisfies equations
\eqref{11.3.1} and \eqref{11.3.2} and hence
is a linear operator.

\ex{11.3.8}
\begin{exenum}
\x
Show that the operation of multiplication by a given
function $f(x)$ is a linear operator.
That is, prove that, if $M$ is defined by
\[
My = f(x)y
,
\]
then $M$ is the linear operator.
\x
Show that the composition of a linear operator $L$
followed by the operation of multiplication by
$f(x)$ is just the operator $f(x)L$ defined in Problem
\exref{11.3.7}.
\end{exenum}

\ex{11.3.9}
(See Problems \exref{11.3.7} and \exref{11.3.8}.)
If $f(x)$ is a differentiable function and if $D$
is the derivative, then both linear operators
$f(x)D$ and $Df(x)$ are examples of
\dt{linear differential operators} more general
than the type discussed in the text.
Show that
\[
xD \ne Dx
,
\]
by applying both sides to the function $y=x$.
Thus the commutative law of multiplication fails.

\ex{11.3.10}
Let $f$ and $g$ be differentiable complex-valued
functions of a real variable.
Show that the ordinary product rule for
differentiation is still valid; i.e., prove that
\[
\ddx (f(x)g(x)) =
\left(\ddx f(x)\right)g(x) +
f(x)\left(\ddx g(x)\right)
.
\]
[\emph{Hint:} Let $f(x) = f_1(x)+if_2(x)$
and $g(x) = g_1(x) + ig_2(x)$, and apply
the definitions of the derivative and of multiplication
of complex numbers.]

\ex{11.3.11}
\begin{exenum}
\x
Let $f$ be a complex-valued function of a real variable
which is differentiable at every point $x$ of an
interval $I$.
Show that if $f^\prime(x)=0$, for every $x$ in $I$,
then $f(x)$ is a constant on $I$.
\x
Let $f$ and $g$ be two complex-valued functions
of a real variable with $f^\prime(x)=g^\prime(x)$
at every point $x$ of some interval $I$.
Show that there exists a complex number $c$
such that $f(x)=g(x)+c$, for every $x$ in $I$.
\end{exenum}

\ex{11.3.12}
Find the general solution of each of the following
differential equations.
\begin{exenum}
\x
$(D-1)^2(D+2)y=0$
\x
$\deriv3y-\deriv2y-4\dydx +4y = 0$.
\end{exenum}

\end{exercises}
