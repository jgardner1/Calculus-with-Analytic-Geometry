\begin{exercises}

\ex{4.3.1}
Evaluate the following definite integrals by finding the limits
of the upper or lower sums.
\begin{exenum}
\x
$\int_0^1 x^2 \; dx$
\x
$\int_0^2 2x \; dx$
\x
$\int_1^3 (x+1) \; dx$
\x
$\int_0^1 (3x^2 + 1) \; dx$.
\end{exenum}

\ex{4.3.2}
For each of the integrals in Problem \exref{4.3.1},
draw the region whose area is given by the integral.

\ex{4.3.3}
Let $f$ be the step function defined by
$f(x) = i$, if $i-1 < x \leq i$,
for every integer $i$.
Draw the graph of $f$ and compute the following integrals.
(\emph{Hint:} These problems are neither hard nor long.
They require an understanding of the definition of
integrability and possibly some ingenuity.
\begin{exenum}
\x
$\int_1^2 f$
\x
$\int_0^3 f$
\x
$\int_{-1}^3 f$
\x
$\int_{-2}^7 f$.
\end{exenum}

\ex{4.3.4}
Every constant function is both increasing and decreasing.
A stronger condition, which excludes constant functions,
is obtained by defining $f$ to be \dt{strictly increasing} if
\[
x<y \quad \mbox{implies  $f(x) < f(y)$}
,
\]
for every $x$ and $y$ in the domain of $f$.
The companion definitions of what it means for a function to be
\dt{strictly decreasing, strictly increasing on an interval,} etc.,
should be obvious.
Using the Mean Value Theorem, prove that if a
differentiable function $f$ satisfies the inequality
$f^\prime(x)>0$ for every $x$ in an interval $I$,
then $f$ is strictly increasing on $I$.

\ex{4.3.5}
Prove the converse of Theorem \thref{4.3.2};
i.e., if $f$ is integrable over $[a,b]$, then
$\lim_{n\goesto\infty} (U_n - L_n) = 0$.
(This is a difficult problem.)

\end{exercises}
