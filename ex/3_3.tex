\begin{exercises}

\ex{3.3.1}
Describe and sketch the graph of each of the following equations.
Label the foci and endpoints of the major and minor axes.
\begin{exenum}
\sx
$\frac{x^2}4 + \frac{y^2}9 = 1$
\sx
$\frac{x^2}9 + \frac{y^2}4 = 1$
\sx
$\frac{x^2}{169} + \frac{y^2}{144} = 1$
\sx
$\frac{x^2}{100} + \frac{y^2}{64} = 1$
\sx
$\frac{x^2}{17} + \frac{y^2}{16} = 1$.
\end{exenum}

\ex{3.3.2}
Write an equation for the ellipse satisfying the given conditions.
\begin{exenum}
\sx
Foci at $(-5,0)$ and $(5,0)$.  Minor axis of length $24$.
\sx
Center at the origin.  Major axis horizontal and of length $14$,
minor axis vertical and of length $8$.
\sx
Center at the origin.  Minor axis vertical and of length $4$.
Passing though $(3,1)$.
\sx
Foci at $(-4,0)$ and $(4,0)$.
Endpoints of major axis at $(-5,0)$ and $(5,0)$.
\sx
The locus of points the sum of whose distances from
$(0,2)$ and $(0,-2)$ is $7$.
\end{exenum}

\ex{3.3.3}
It has been shown that the distance between a point on
the ellipse
\[
\frac{x^2}{a^2} + \frac{y^2}{a^2-c^2} = 1
\]
and the focus $(c,0)$ is
$\left| \frac{xc}a - a \right|$.
\begin{exenum}
\sx
Show that this distance is $a - \frac{xc}a$ for
$|x| \leq a$.
\sx
Show that the distance between a point on the ellipse
and the focus $(-c,0)$ is $\left| \frac{xc}a + a \right|$
and that the distance is $a + \frac{xc}a$.
\sx
Show that the sum of the distances from a point on the
ellipse to the foci is $2a$ and hence that the graph of
$\frac{x^2}{a^2} + \frac{y^2}{a^2-c^2} = 1$
contains only those points which satisfy the locus definition.
\end{exenum}

\ex{3.3.4}
Describe and sketch the graph of each of the following equations.
\begin{exenum}
\sx
$\frac{(x-2)^2}{25} + \frac{(y-4)^2}{9} = 1$
\sx
$\frac{(x+3)^2}{16} + \frac{(y-2)^2}{25} = 1$
\sx
$\frac{(x+5)^2}{169} + \frac{(y+2)^2}{144} = 4$
\sx
$25x^2 + 9(y+3)^2 = 225$
\sx
$9x^2 + 4y^2 + 36x - 24y + 36 = 0$.
\end{exenum}

\ex{3.3.5}
The line segment which passes though a focus,
is perpendicular to the major axis,
and has its endpoints on the ellipse is called a
\dt{latus rectum}.
\begin{exenum}
\sx
Find the length of a latus rectum of the ellipse
$4x^2 + 9y^2 = 36$.
\sx
Find the length of a latus rectum of the ellipse
$b^2x^2 + a^2y^2 = a^2b^2$.  (Assume that $b < a$.)
\sx
Show that both latera recta of an ellipse are the same length.
\end{exenum}

\ex{3.3.6}
Write equations of the directrices and find the eccentricity
of each of the following ellipses.
\begin{exenum}
\sx
$4x^2 + 9y^2 = 36$
\sx
$9x^2 + 4y^2 = 144$.
\end{exenum}

\ex{3.3.7}
Assume that $0 < c < a$.
\begin{exenum}
\sx
\sxlab{3.3.7a}
Find the distance between $(x,y)$ and $(-c,0)$.
\sx
\sxlab{3.3.7b}
Find the distance between $(x,y)$ and the line $x = -\frac{a^2}c$.
\sx
Find the locus of points $(x,y)$ such that the ratio between
the distance in \exref{3.3.7a} and the distance in \exref{3.3.7b}
is a constant $\frac ca$.
\end{exenum}

\ex{3.3.8}
Show that an ellipse becomes more nearly circular
as its foci get closer and closer together.

\ex{3.3.9}
Consider a point $(x_1,y_1)$ on the graph of
$b^2x^2 + a^2y^2 = a^2b^2$.
\begin{exenum}
\sx
\sxlab{3.3.9a}
Find the slope of the tangent to the graph at $(x_1,y_1)$.\
\sx
Write an equation of the tangent line in \exref{3.3.9a}.
\sx
Show that $b^2xx_1 + a^2yy_1 = a^2b^2$
is an equation of the tangent line.
\end{exenum}

\ex{3.3.10}
Assume that the constants $a$, $b$, $c$, $d$, and $e$
are such that $ax^2 + by^2 + cx + dy + e = 0$
is an equation of an ellipse.
Consider a point $(x_1,y_1)$ on this ellipse.
\begin{exenum}
\sx
\sxlab{3.3.10a}
Find the slope of the tangent to the graph at $(x_1,y_1)$.
\sx
Write an equation of the tangent line in \exref{3.3.10a}.
\sx
Show that $axx_1 + byy_1 + \frac12c(x+x_1) +
\frac12d(y+y_1) + e = 0$ is an equation of the tangent line.
\end{exenum}

\ex{3.3.11}
Write an equation of the ellipse with horizontal and vertical
axes satisfying the given data.
\begin{exenum}
\sx
Foci at $(-3,2)$ and $(5,2)$.  Eccentricity is $\frac23$.
\sx
Center at $(2,-1)$.  One focus at $(2,2)$.
Point $(2,4)$ lies on ellipse.
\sx
Ends of major axis at $(2,4)$ and $(12,4)$.
Ends of minor axis at $(7,2)$ and $(7,6)$.
\end{exenum}

\ex{3.3.12}
In the definition of the ellipse, we asserted that the
constant must be greater than the distance between the foci.
What is the locus of points in the plane the sum of whose
distances from $(-c,0)$ and $(c,0)$ is the constant $2c$?

\ex{3.3.13}
What must be the relation between $a$ and $b$ for the
major axis of the ellipse
$\frac{x^2}{a^2} + \frac{y^2}{b^2} = 1$
to be horizontal?  Vertical?

\end{exercises}
