\begin{exercises}

\ex{3.1.1}
Write an equation for each of the following.
\begin{exenum}
\sx
A circle with center at the origin and radius $3$.
\sx
A circle with center at the origin and radius $\frac{10}3$.
\sx
A circle with center at $(-2, 2)$ and radius $5$.
\sx
A circle with center at $(3, 0)$ and radius $3$.
\sx
A circle with center at $(0, -7)$ and radius $7$.
\sx
A circle with center at $(-3, -3)$ and radius $3\sqrt2$.
\sx
A circle with center in the first quadrant, radius $4$,
and tangent to both axes.
\sx
A circle with center on the $y$-axis, radius $\frac52$,
and tangent to the $x$-axis (there are two such circles).
\sx
A circle with radius $2$ and tangent to the $x$-axis
and to the line defined by the equation $x=5$
(there are four such circles).
\end{exenum}

\ex{3.1.2}
For each of the following equations,
describe the curve defined by it.
\begin{exenum}
\sx
$x^2 + y^2 = 64$
\sx
$x^2 + y^2 = 32$
\sx
$x^2 + (y-4)^2 = 9$
\sx
$(x+2)^2 + y^2 = 16$
\sx
$(x-2)^2 + (y+7)^2 = 19$
\sx
$(2x-3)^2 + (2y-5)^2 = \frac{25}4$
\sx
$x^2 + y^2 - 8x - 12y + 27 = 0$
\sx
$x^2 + y^2 - 5x - 7y + \frac52 = 0$
\sx
$9x^2 + 9y^2 - 12x + 30y = 71$
\sx
$5x^2 + 5y^2 - 6x + 8y = 31$.
\end{exenum}

\ex{3.1.3}
Show that, if $b^2 + c^2 > 4ad$ and $a \ne 0$, the equation
$ax^2 + ay^2 + bx + cy + d = 0$ defines a circle with
center at
\[
\left(-\frac{b}{2a}, -\frac{c}{2a} \right) 
\mbox{and radius}
\sqrt{\frac{b^2+c^2-4ad}{4a^2}}
.
\]

\ex{3.1.4}
Show that the tangents from a point $(x_1,y_1)$
outside the circle defined by
$(x-h)^2 + (y-k)^2 = r^2$ to the circle are of length
\[
\sqrt{(x_1-h)^2 + (y_1-k)^2 - r^2}
.
\]

\ex{3.1.5}
Show that the line tangent to the circle defined by
\[
ax^2 + ay^2 + bx + cy + d = 0 \mbox{at $(x_1,y_1)$
has the equation}
\]
\[
ax_1x + ay_1y + \frac b2(x+x_1) + \frac c2 (y+y_1) + d = 0
.
\]

\ex{3.1.6}
Write an equation of the line which is tangent to
\begin{exenum}
\sx
the circle defined by $x^2 + y^2 = 25$ at $(-3,4)$.
\sx
the circle defined by $x^2 + y^2 = 9$ at $(0,3)$.
\sx
the circle defined by $(x-2)^2 + (y+8)^2 = 169$ at $(7,4)$.
\sx
the circle defined by $x^2 + y^2 - 10y = 33$ at $(7,2)$.
\end{exenum}

\ex{3.1.7}
Write an equation of the line containing the common chord
of circles defined by
$x^2 + y^2 - 8x - 12y = 48$ and
$x^2 + y^2 - 4x + 6y = 23$.

\ex{3.1.8}
Given a circle and a line tangent to it,
the segment of the line between a given point and the point
of tangency is commonly called the tangent from the point
to the circle.  Show that the locus of points from which the
tangents to two unequal externally tangent circles have equal
length is the common internal tangent line.

\ex{3.1.9}
Write an equation for the circle which passes through
\begin{exenum}
\sx
$(3,4)$, $(-4,3)$, and $(5,0)$.
\sx
$(7,1)$, $(6,2)$, and $(-1,-5)$.
\sx
$(4,16)$, $(-6,-8)$, and $(11,9)$.
\end{exenum}

\ex{3.1.10}
Use the results of Problems \exref{3.1.3} and \exref{3.1.5}
to show that a line tangent to a circle is perpendicular
to the radius drawn to the point of tangency.

\end{exercises}
