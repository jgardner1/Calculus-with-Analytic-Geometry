\begin{exercises}

\ex{2.2.1}
Generalize on Example \exampref{2.2.1} to show that the largest
rectangle with a fixed perimeter $p$ is a square
with side $\frac p4$.

\ex{2.2.2}
A field is bounded on one side by a stone wall.
A rectangular plot of ground is to be fenced off, using
the stone wall as one boundary and $200$ yards of
fencing for the other three sides.
What are the dimensions of the largest such plot?

\ex{2.2.3}
Find the positive number which is such that the sum of the number
and its reciprocal is a minimum.

\ex{2.2.4}
List all local extreme points and all absolute extreme points
for each of the following functions, noting carefully its domain
of definition.  Classify each extreme point by type.
\begin{exenum}
\sx
$3x^5 - 5x^3 + 7$; domain: all real numbers.
\sx
$4x^3 + 3x^2 - 6x + 5$; domain: all real numbers.
\sx
$x + \frac{a^2}x$; domain: all nonzero numbers.
\sx
$2x^3 - 21x^2 + 60x - 25$; domain: all nonnegative real numbers.
\sx
$\frac{x^2}{x-1}$; domain: all real numbers except $1$.
\sx
$3x^4 - 20x^3 - 36x^2 + 54$; domain: all nonpositive real numbers.
\sx
$(x-1)^2(x+1)^3$; domain: all nonnegative real numbers
no greater than $2$.
\sx
$2-(x+4)^\frac23$; domain: all real numbers.
\sx
$(x-1)^2(x-4)$; domain: all nonnegative real numbers. 
\end{exenum}

\ex{2.2.5}
Generalize on Example \exampref{2.2.2} to show that the right circular
cylinder with a fixed volume and the least total surface area
has a diameter equal to its height.

\ex{2.2.6}
Show that $f(x)=x^4$ has an extreme point where the second
derivative is neither positive nor negative.  What type of extreme
point is it?  Explain why this is not a contradiction of Theorem
\thref{2.2.2}.

\ex{2.2.7}
A line has positive intercepts on both axes and their sum is $8$.
Write an equation of the line if it cuts off in the first quadrant
a triangle with area as large as possible.

\ex{2.2.8}
Find two nonnegative numbers, $x$ and $y$, such that
$x+y=6$ and $x^2y$ is as large as possible.

\ex{2.2.9}
Find all ordered pairs, $(x, y)$, such that $xy=9$ and
$\sqrt{x^2 + y^2}$ is a minimum.
Interpret your result geometrically.

\ex{2.2.10}
\begin{exenum}
\sx
\sxlab{2.2.10a}
Graph the set of ordered pairs $(x,y)$ such that
$4x^2 + y^2 = 8$.  The graph is called an ellipse.
\sx
Find all ordered pairs $(x,y)$, such that $4x^2 + y^2 = 8$
and $4xy$ is a maximum.
\sx
Find the dimensions of the largest (in area) rectangle which
has sides parallel to the $x$-axis and the $y$-axis and is
inscribed in the ellipse of \exref{2.2.10a}.
\end{exenum}

\ex{2.2.11}
Find the dimensions of the largest rectangle which has its upper
two vertices on the $x$-axis and the other two on the graph of
$y = x^2 - 27$.

\ex{2.2.12}
Find the dimensions of the rectangle which has its upper two vertices
on the $x$-axis and the other two on the graph of $y = x^2 - 27$
and which has maximum perimeter.

\ex{2.2.13}
A box without a top is to be made by cutting equal squares from the
corners of a rectangular piece of tin $30$ inches by $48$ inches and
bending up the sides.  What size should the squares be if the volume
of the box is to be a maximum?  [\emph{Hint:} If $x$ is the side
of a square, $V(x) = x(30-2x)(48-2x)$.]

\ex{2.2.14}
\begin{exenum}
\sx
\sxlab{2.2.14a}
A box without a top is to be made by cutting equal squares from
the corners of a square piece of tin, $18$ inches on a side,
and bending up the sides.  How large should the squares be
if the volume of the box is to be as large as possible?
\sx
Generalize \exref{2.2.14a} to the largest open-topped box which
can be made from a square piece of tin, $s$ inches on a side.
\end{exenum}

\ex{2.2.15}
\begin{exenum}
\sx
Where should a wire $20$ inches long be cut if one piece is to be
bent into a circle, the other piece is to be bent into a square,
and the two plane figures are to have areas the sum of which
is a maximum?
\sx
Where should the cut be if the sum of areas is to be a minimum?
\end{exenum}

\ex{2.2.16}
A man in a canoe is $6$ miles from the nearest point of the shore
of the lake.  The shoreline is approximately a straight line and the man wants
to reach a point on the shore $5$ miles from the nearest point.
If his rate of paddling is $4$ miles per hour and he can run $5$
miles per hour along the shore, where should he land to reach his
destination in the shortest possible time?

\ex{2.2.17}
Prove that the largest isosceles triangle which can be inscribed
in a given circle is also equilateral.

\ex{2.2.18}
Prove that the smallest isosceles triangle which can be circumscribed
about a given circle is also equilateral.

\ex{2.2.19}
What is the smallest positive number that can be written as the sum of
two positive numbers $x$ and $y$ so that $\frac1x + \frac2y = 1$?

\ex{2.2.20}
An excursion train is to be run for a lodge outing.  The railroad company
sets the rate at $\$10$ per person if less that $200$ tickets are sold.
They agree to lower the rate per person by $2$ cents for each ticket
sold above the $200$ mark, but the train will only hold $450$ people.
What number of tickets will give the company the greatest income?

\ex{2.2.21}
Consider the continuous real-valued function $f(x) = x$ with domain
$0 < x < 1$.  Does this function have an absolute maximum point
or an absolute minimum point?
Why is this function not a counterexample to Theorem \thref{2.2.4}?

\ex{2.2.22}
Let
\[
f(x) = \dilemma{\frac1x, & for\dilemma{-1 \leq x < 0,}{0 < x \leq 1,}}
{0, & \mbox{for $x = 0$}.}
\]
This real-valued function is defined on the closed interval $[-1,1]$.
Draw the graph of $f(x)$ and explain why this function has neither
absolute maximum nor absolute minimum points.

\end{exercises}
