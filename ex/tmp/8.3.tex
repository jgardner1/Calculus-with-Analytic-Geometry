\begin{exercises}

\ex{8.3.1}
Use the Midpoint Rule with $n=4$ to compute
approximations to the following integrals.
In \exref{8.3.1a}, \exref{8.3.1b}, \exref{8.3.1c},
\exref{8.3.1d}, and \exref{8.3.1e} compare
the result obtained with the true value.
\begin{exenum}
\x
\xlab{8.3.1a}
$\int_0^1 (x^2+1)\;dx$
\x
\xlab{8.3.1b}
$\int_{-1}^3 (6x-5)\;dx$
\x
\xlab{8.3.1c}
$\int_1^3 \frac1{x^2} dx$
\x
\xlab{8.3.1d}
$\int_0^3 \frac1{1+x} dx$
\x
\xlab{8.3.1e}
$\int_0^3 \sqrt{1+x}\;dx$
\x
$\int_0^{2\pi} \sin^2x\;dx$
\x
$\int_0^1 e^{-x^2} dx$
\x
\xlab{8.3.1h}
$\int_0^1 \sqrt{1+x^3}\; dx$.
\end{exenum}

\ex{8.3.2}
\exref{8.3.1a} through \exref{8.3.1h}.
Compare $M_4$, the Midpoint Approximation
computed in Problem \exref{8.3.1}, to
$T_4$, the corresponding
Trapezoid Approximation.

\ex{8.3.3}
\exref{8.3.1a} through \exref{8.3.1h}.
Use Simpson's Rule with $n=4$ to compute
an approximation to the corresponding definite
integral in Problem \exref{8.3.1}.

\ex{8.3.4}
Show geometrically that, if the graph of $f$
is concave up at every point of the interval
$[a,b]$, then the Midpoint Approximation
is too small and the Trapezoid Approximation
is too big; i.e.,
\[
M_n < \int_a^b f < T_n
.
\]

\ex{8.3.5}
Do Problem \exref{8.3.4} analytically
by using the remainder formulas
\thref{8.2.4} and \thref{8.3.3}.

\ex{8.3.6}
Show that Simpson's Approximation is the
weighted average of the Trapezoid Approximation
and the Midpoint Approximation.
Specifically, for any even positive integer
$n=2m$, show that
\[
S_n = \frac13T_m + \frac23 M_m
.
\]

\ex{8.3.7}
Prove Theorem \thref{8.3.5}, i.e.,
if $f$ is integrable over $[a,b]$, then
\[
\lim_{n\goesto\infty} S_n = \int_a^b f
,
\]
by showing that it is a direct corollary of the
result of Problem \exref{8.3.6} and the two
corresponding theorems,
$\lim_{n\goesto\infty}T_n=\int_a^b f$ and
$\lim_{n\goesto\infty}M_n=\int_a^b f$.

\ex{8.3.8}
For each of the following integrals and each of
the three methods of numerical integration
(Trapezoid Rule, Midpoint Rule, and
Simpson's Rule), find the smallest integer $n$
such that the error obtained is less that
$10^{-4}$.
As the basis for finding $n$, use Theorems
\thref{8.2.4}, \thref{8.3.3}, and \thref{8.3.6}.
\begin{exenum}
\x
$\int_1^4 \left(\frac1{2x^2} + \frac{x^2}2\right)\;dx$
\x
$\int_0^2 \frac1{2x+1} dx$.
\end{exenum}

\ex{8.3.9}
This problem is analogous to Problem \exref{8.3.6}.
Show that, for any positive integer $n$,
\[
T_{2n} = \frac12T_n + \frac12M_n
.
\]

\ex{8.3.10}
Suppose that the graph of $f$ is concave
up at every point of the interval $[a,b]$.
\begin{exenum}
\x
Using the results of Problems \exref{8.3.4}
and \exref{8.3.9}, show that
\[
T_{2n} - (T_n - T_{2n}) < \int_a^b f<T_{2n}
,
\]
for every positive integer $n$.
\x
\xlab{8.3.10b}
Hence show that the error $|\int_a^b f-T_{2n}|$
in the Trapezoid Approximation satisfies
\[
\left| \int_a^b f-T_{2n}\right|
< |T_n - T_{2n}|
.
\]
\x
Show that \exref{8.3.10b} also holds if the graph
of $f$ is concave down at every point of
$[a,b]$.
\end{exenum}

\end{exercises}
