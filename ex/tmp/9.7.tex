\begin{exercises}

\ex{9.7.1}
Let $f$ be the function defined by
\[
f(x) = \sum_{i=1}^\infty \frac{x^i}{i^2}.
\]
\begin{exenum}
\x
Find the radius of convergence of the power series
and also the domain of $f$.
\x
Write the derived series, and find its radius of
convergence directly
[i.e., verify Theorem \thref{9.7.1} for
this particular series].
\x
Find the domain of the function defined
by the derived series.
\end{exenum}

\ex{9.7.2}
Let $f$ be the function defined by
$f(x) = \sum_{i=1}^\infty \frac1{i2^i} (x-2)^i$,
and follow the same instructions as in
Problem \exref{9.7.1}.

\ex{9.7.3}
Let $f$ be the function defined by
$f(x) = \sum_{i=1}^\infty \frac1{\sqrt{i}} (x-2)^i$,
and follow the same instructions as in
Problem \exref{9.7.1}.

\ex{9.7.4}
Let $f$ be the function defined by
$f(x) = \sum_{i=0}^\infty \frac{(x-1)^i}{\sqrt{i+1}}$,
and follow the same instructions as in
Problem \exref{9.7.1}.

\ex{9.7.5}
Find the domains of the functions $f$ and $g$
defined by the following power series.
\begin{exenum}
\x
$f(x) = \sum_{i=1}^\infty \frac{(-1)^{i-1}}
{(2i-1)!} x^{2i-1}$
\x
$g(x) = \sum_{i=0}^\infty \frac{(-1)^{i}}
{(2i)!} x^{2i}$.
\end{exenum}

\ex{9.7.6}
If $f$ and $g$ are the two functions defined
in Problem \exref{9.7.5}, show that
\begin{exenum}
\x
$f^\prime (x) = g(x)$
\x
$g^\prime (x) = - f(x)$.
\end{exenum}

\ex{9.7.7}
Let $f$ and $g$ be the two functions defined in
Problem \exref{9.7.5} (see also Problem \exref{9.7.6}).
\begin{exenum}
\x
\xlab{9.7.7a}
Evaluate $f(0)$, $f^\prime(0)$, $g(0)$,
and $g^\prime(0)$.
\x
\xlab{9.7.7b}
Show that $f$ and $g$ are both solutions of the
differential equation $\frac{d^2y}{dx^2} + y = 0$.
\x
Write the general solution of the differential equation
in \exref{9.7.7b}, and thence, using the results of
part \exref{9.7.7a}, show that $f(x) = \sin x$
and that $g(x) = \cos x$.
\end{exenum}

\ex{9.7.8}
Show as is claimed at the beginning of the proof
of Theorem \thref{9.7.2}, that it is a direct
consequence of the Chain Rule that if this
theorem is proved for $a=0$, then it is
true for an arbitrary real number $a$.

\ex{9.7.9}
Prove that every power series can be integrated,
term by term.  Specifically,
prove the following two theorems.
\begin{exenum}
\x
\emph{A power series
$\sum_{i=0}^\infty a_i(x-a)^i$ and its integrated
series
\[
\sum_{i=0}^\infty \frac{a_i}{i+1} (x-a)^{i+1}
\]
have the same radius of convergence.}
\x
\emph{If the radius of convergence $\rho$
of the power series
$\sum_{i=0}^\infty a_i(x-a)^i$
is not zero and if $f$ and $F$ are the functions
defined, respectively, by
\[
f(x) = \sum_{i=0}^\infty a_i(x-a)^i \quad \mbox{and}
\quad F(x) = \sum_{i=0}^\infty \frac{a_i}{i+1} (x-a)^{i+1}
,
\]
then
\[
F(x) = \int f(x) \; dx + c
.
\]
}
\end{exenum}

\ex{9.7.10}
Starting from the geometric series
\[
\sum_{i=0}^\infty (-1)^ix^{2i} =
1 - x^2 + x^4 - x^6 + \cdots
\]
and the results of Problem \exref{9.7.9}, show that
\[
\arctan x = \sum_{i=0}^\infty
\frac{(-1)^i}{2i+1} x^{2i+1}
,
\]
for every $x$ such that $|x| < 1$.

\ex{9.7.11}
\begin{exenum}
\x
Show that the interval of convergence of the
integrated series in Problem \exref{9.7.10}
is the closed interval $[-1,1]$.
\x
True or false?
\[
\frac{\pi}4 = \arctan 1 = \arctan x = \sum_{i=0}^\infty
\frac{(-1)^i}{2i+1}
.
\]
\end{exenum}

\end{exercises}
