\begin{exercises}

\ex{6.1.1}
Find the values of
\begin{exenum}
\x
$\sin(-\pi)$ and $\cos(-\pi)$
\x
$\sin(\frac34\pi)$ and $\cos(\frac34\pi)$
\x
$\sin(-\frac{\pi}2)$ and $\cos(-\frac{\pi}2)$
\x
$\sin(\frac{\pi}6)$ and $\cos(\frac{\pi}6)$
\x
$\sin(\frac{\pi}3)$ and $\cos(\frac{\pi}3)$
\x
$\sin(\frac{5\pi}4)$ and $\cos(\frac{5\pi}4)$.
\end{exenum}

\ex{6.1.2}
Make a table like the one below showing the sign
of $\cos t$ and $\sin t$ in each of the four quadrants.
Put $+$ or $-$ in each entry of the table.
\[
\mbox{TABLE, PLEASE!}
\]

\ex{6.1.3}
Find all values of $t$ such that
\begin{exenum}
\x
$\sin t = 0$
\x
$\cos t = 0$
\x
$\sin t = 1$
\x
$\cos t = 1$
\x
$\sin t = -1$
\x
$\cos t = -1$
\x
$\sin t = 2$
\x
$\cos t = 2$.
\end{exenum}

\ex{6.1.4}
What is the domain and range of each of the
functions of $\cos$ and $\sin$?

\ex{6.1.5}
If $k$ is an arbitrary integer, find
\begin{exenum}
\x
$\cos k\pi$
\x
$\sin k\pi$
\x
$\cos \left( \frac{\pi}2 + k\pi \right)$
\x
$\sin \left( \frac{\pi}2 + k\pi \right)$.
\end{exenum}

\ex{6.1.6}
Remembering that
$\frac{\pi}4 + \frac{\pi}6 = \frac{5\pi}{12}$ and
$\frac{\pi}4 - \frac{\pi}6 = \frac{\pi}{12}$, find
\begin{exenum}
\x
$\sin \frac{5\pi}{12}$
\x
$\cos \frac{5\pi}{12}$
\x
$\sin \frac{\pi}{12}$
\x
$\cos \frac{\pi}{12}$.
\end{exenum}

\ex{6.1.7}
If $f$ is a function with the property that
$f(t+2\pi) = f(t)$, for every real number
$t$, show from this that
\begin{exenum}
\x
$f(t-2\pi) = f(t)$, for every real number $t$.
\x
$f(t+2\pi n) = f(t)$, for every real number $t$
and every integer $n$.
(Use induction.)
\end{exenum}

\ex{6.1.8}
\begin{exenum}
\x
\xlab{6.1.8a}
Use \thref{6.1.5} to write a formula for
$\cos 2a$ in terms of $\cos a$ and $\sin a$.
\x
Similarly, use \thref{6.1.7} to write a formula
for $\sin 2a$.
\x
Write a formula for $\cos a$ and another
for $\sin a$ in terms of
$\cos \frac a2$ and $\sin \frac a2$.
\end{exenum}

\ex{6.1.9}
Use the formula for $\cos 2a$ [Problem \exref{6.1.8a}]
and identity $1 = \cos^2a + \sin^2a$
to derive a formula for
\begin{exenum}
\x
$\cos^2a$ in terms of $\cos 2a$
\x
$\sin^2a$ in terms of $\cos 2a$.
\end{exenum}

\ex{6.1.10}
Let $f$ be a function which is periodic with period
$2\pi$, i.e., $f(t+2\pi) = f(t)$,
and suppose that the graph of $f$ for
$0 \leq t \leq 2\pi$ is as shown in Figure \figref{6.5}.
Draw the graph of $f$ for
$-2\pi \leq t \leq 6\pi$.

\end{exercises}
