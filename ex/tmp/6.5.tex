\begin{exercises}

\ex{6.5.1}
In each of the following examples identify the
function $f$ as a polynomial or not.
If it is not a polynomial, give a reason.
(Consider such things as the vanishing of
higher-order derivatives, or the behavior of $f$,
or of some derivative $f^{(j)}$,
near a point of discontinuity.)
\begin{exenum}
\x
$f(x) = \frac1x$
\x
$f(x) = \frac{x-1}{x+1}$
\x
$f(x) = \pi x^2 + ex + 2$
\x
$f(x) = x^{\frac23} + x^{\frac13}$
\x
$f(x) = \sqrt{x^2-2}$
\x
$f(x) = x(x^2-7)$
\x
$f(x) = e^x$
\x
$f(x) = \tan x$.
\end{exenum}

\ex{6.5.2}
Prove that the algebraic function $g$ defined by
$g(x) = \sqrt{x^3+2}$ is not rational.
[\emph{Hint:} Suppose it is rational.
Then there exist polynomials $p$ and $q$ such that
$\sqrt{x^3+2} = \frac{p(x)}{q(x)}$,
for every $x \geq -\sqrt[3]2$.  But then
\[
x^3+2 = \left[ \frac{p(x)}{q(x)} \right]^2
,
\]
or, equivalently,
\[
(x^3+2)[q(x)]^2 - [p(x)]^2 = 0,
\cond{\mbox{for all $x \geq -\sqrt[3]2$.}}
\]
The left side of this equation is a polynomial which
is not identically zero.
(Why?)
How many roots can such a polynomial have?]

\ex{6.5.3}
Prove that each of the following functions $f$
is algebraic by exhibiting a polynomial
$F(x,y)$ and showing that $F(x,f(x)) = 0$.
\begin{exenum}
\x
$f(x) = \sqrt{\frac{x+1}{x-1}}$
\x
$f(x) = \ddx \arctan x$
\x
$f(x) = \ddx \arcsin x$
\x
$f(x) = \ddx \arcsec x$
\x
$f(x) = \ln 5^x$
\x
$f(x) = 2x + \sqrt{4x^2-1}$.
\end{exenum}

\end{exercises}
