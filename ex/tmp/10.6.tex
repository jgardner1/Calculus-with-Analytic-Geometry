\begin{exercises}

\ex{10.6.1}
\begin{exenum}
\x
\xlab{10.6.1a}
For each of the following values of $\theta$, find
the value of $r$ such that $r=4 \sin \theta$:
\[
\theta = 0, \frac\pi6, \frac\pi4, \frac\pi2,
\frac{3\pi}4, \frac{5\pi}6, \pi
.
\]
\x
Plot the seven points with the polar coordinates
$(r,\theta)$ found in part \exref{10.6.1a}.
\x
Draw and identify the curve defined by the equation
$r=4\sin\theta$ in polar coordinates.
\end{exenum}

\ex{10.6.2}
\begin{exenum}
\x
\xlab{10.6.2a}
For each of the following values of $\theta$,
find the value of $r$ such that
$r = 2(1+\cos \theta)$:
\[
\theta = 0, \frac\pi4, \frac\pi3, \frac\pi2, \frac{2\pi}3,
\frac{5\pi}6, \pi
.
\]
\x
Plot the seven points with the polar coordinates
$(r,\theta)$, found in part \exref{10.6.2a}.
\x
\xlab{10.6.2c}
What symmetry property is possessed by the
curve defined by the equation
$r=2(1+\cos\theta)$ in polar coordinates?
\x
Draw the curve in part \exref{10.6.2c}.
\end{exenum}

\ex{10.6.3}
Using a figure and the geometric interpretation
of polar coordinates, show that
$r = \frac5{\cos\theta}$ is an equation in polar
coordinates of the vertical line cutting the
$x$-axis in the point $(5,0)$.

\ex{10.6.4}
Using a figure and the geometric interpretation
of polar coordinates, find an equation in polar
coordinates of the horizontal line cutting
the $y$-axis in the point $(0,5)$.

\ex{10.6.5}
Assume the well-known fact that,
if one side of a triangle inscribed in a circle
is a diameter, then the triangle is a right triangle.
Using this fact and the geometric interpretation
of polar coordinates, show that
$\cos\theta = \frac{r}{2a}$ is an equation
of the circle which passes through the origin
and has radius $a>0$ and center on
the $x$-axis.

\ex{10.6.6}
Identify and draw the polar graphs of
the two equations
\begin{exenum}
\x
$r=7$
\x
$\theta = \frac\pi6$.
\end{exenum}

\ex{10.6.7}
Consider the curves defined by each of the following
equations in polar coordinates.
Write each curve as the graph of an equation
in $x$- and $y$-coordinates.
Identify and draw the curve in the $xy$-plane.
\begin{exenum}
\x
$r\cos\theta = -2$
\x
$r\sin\theta = 4$
\x
$r=-4\cos\theta$
\x
$r=\frac2{\sin\theta-2\cos\theta}$
\x
$r=\frac1{1-\cos\theta}$ (see Example \exampref{10.6.3})
\x
$r=5$
\x
$\theta = \arcsin \frac3{\sqrt{10}}$
\x
$r = \frac1{2-\sqrt3 \cos\theta}$.
\end{exenum}

\ex{10.6.8}
Let $f$ be a real-valued function of a real variable.
Prove that:
\begin{exenum}
\x
If $f$ is an even function, then the polar graph
of the equation $r=f(\theta)$ is symmetric
about the $x$-axis.
\x
If $f$ is an odd function, then the polar graph
of the equation $r = f(\theta)$ is symmetric
about the $y$-axis.
\end{exenum}

\ex{10.6.9}
Let $F$ be a real-valued function of two real variables.
Prove that the polar graph of the equation
$F(r^2,\theta) = 0$ is symmetric about the origin.

\ex{10.6.10}
Draw the curve defined by each of the following
equations in polar coordinates (the number
$a$ is an arbitrary positive constant).
\begin{exenum}
\x
$r=a(1+\cos\theta)$ (a \dt{cardioid}).
\x
$r=a(2+\cos\theta)$ (a \dt{lima\c{c}on}).
\x
$r=a(\frac12 + \cos\theta)$ (a \dt{lima\c{c}on}).
\x
$r^2 = 2a^2\sin2\theta$ (a \dt{lemniscate}).
\x
$r\theta = 2$ (a \dt{hyperbolic spiral}).
\end{exenum}

\ex{10.6.11}
Consider the Archimedean spiral defined by
the equation $r=a\theta$ and discussed in
Example \exampref{10.6.4}.
Describe the space of tangent vectors to this
curve at $\theta = 0$, and also at $\theta = \frac\pi2$.

\ex{10.6.12}
\begin{exenum}
\x
\xlab{10.6.12a}
Show that the equations $y=4\cos x$ and
$y^2=4y\cos x$ are not equivalent.
\x
In spite of part \exref{10.6.12a}, the polar graphs
of $r=4\cos\theta$ and of $r^2 = 4r\cos\theta$
are the same.  Explain.
\end{exenum}

\ex{10.6.13}
\begin{exenum}
\x
If $f$ is a real-valued function of a real variable,
prove that the polar graph of the equation
$r=f(\sin\theta)$ is symmetric about the $y$-axis.
\x
Draw the curve (a cardioid) defined by the equation
$r=2(1+\sin\theta)$ in polar coordinates.
\x
Draw the curve (a lima\c{c}on) defined by the
equation $r=1+2\sin\theta$ in polar
coordinates.
\end{exenum}

\end{exercises}
