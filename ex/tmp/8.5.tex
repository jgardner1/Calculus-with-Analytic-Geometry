\begin{exercises}

\ex{8.5.1}
Compute the work in foot-pounds done by
the force of gravity when a $50$-pound
rock falls $200$ feet off a vertical cliff.

\ex{8.5.2}
Compute the work in foot-pounds done against
the force of gravity in raising a $10$-pound weight
vertically $6$ feet from the ground.

\ex{8.5.3}
A car on a horizontal track is attached to a fixed
point by a spring, as shown in Figure
\figref{8.24}.  The spring has been stretched
$2$ feet beyond its rest position, and the
car is held there by a force of $10$ pounds.
If the car is released, how many foot-pounds
of work are done by the restoring force of
the spring moving the car $2$ feet back
to the rest position?

\ex{8.5.4}
An electron is attracted to a nucleus by a
force which is inversely proportional to the
square of the distance $r$ between them; i.e.,
$\frac{k}{r^2}$.  If the nucleus is fixed,
compute the work done by the attractive
force in moving the electron from $r=2a$
to $r=a$.

\ex{8.5.5}
A container holding water is raised vertically
a distance of $10$ feet at the constant
rate of $10$ feet per minute.
Simultaneously water is leaking from the container
at the constant rate of 15 pounds per minute.
If the empty container weighs $1$ pound
and if it holds $15$ pounds of water at the beginning
of the motion, find the work done against
the force of gravity.

\ex{8.5.6}
Suppose that a straight cylindrical hole is bored
from the surface of the earth through the center
and out the other side.
An object of mass $m$ inside the hole and at a
distance $r$ from the center of the earth is
attracted to the center by a gravitational
force equal in absolute value to $\frac{mgr}{R}$,
where $g$ is constant and $R$ is the radius
of the earth.  Compute the work done by this
force of gravity in terms of
$m$, $g$, and $R$ as the object falls
\begin{exenum}
\x
from the surface to the center of the earth,
\x
from the surface of the earth through the center
to a point halfway between the center and surface
on the other side,
\x
all the way through the hole from surface to surface.
\end{exenum}
[\emph{Hint:} Let the $x$-axis be the axis of the
cylinder, and the origin the center of the earth.
Define the gravitational force $F(x)$ acting on the
object at $x$ so that:
(i) its absolute value agrees with the above
prescription, and (ii) its sign agrees with the
convention given at the beginning of \secref{8.5}.]

\ex{8.5.7}
Consider a cylinder and piston as shown in
Figure \figref{8.25}.  The inner chamber,
which contains gas, has a radius $a$ and
lenght $b$.  According to the simplest gas law,
the expansive force of the gas on the piston
is inversely proportional to the volume $v$ of gas;
i.e., $F = \frac{k}{v}$ for some constant $k$.
Compute the work done against this force
in compressing the gas to half its initial
volume by pushing in the piston.

\ex{8.5.8}
A rocket of mass $m$ is on its way from the earth
to the moon along a straight line joining their centers.
Two gravitational forces act simultaneously on the
rocket and in opposite directions.
One is the gravitational pull toward earth,
equal in absolute value to $\frac{GM_1m}{{r_1}^2}$,
where $G$ is the universal gravitational constant,
$M_1$ the mass of the earth, and $r_1$ the
distance between the rocket and the center of
the earth.  The other is the analogous gravitational
attraction toward the moon,
equal in absolute value to $\frac{GM_2m}{{r_2}^2}$,
where $M_2$ is the mass of the moon and $r_2$
is the distance between the rocket and the center
of the moon.  Denote the radii of the earth
of the earth and moon by $a$ and $b$,
respectively, and let $d$ be the distance between
their centers.
\begin{exenum}
\x
Take the path of the rocket for the $x$-axis
with the centers of earth and moon at $0$
and $d$, respectively, and compute $F(x)$,
the resultant force acting on the rocket at $x$.
\x
Set up the definite integral for the work done
against the force $F$ as the rocket moves
from the surface of the earth to the
surface of the moon.
\end{exenum}

\end{exercises}
