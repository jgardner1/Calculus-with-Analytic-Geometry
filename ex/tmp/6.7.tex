\begin{exercises}

\ex{6.7.1}
Write each of the following complex numbers
in the exponential form $|z|e^{it}$.
\begin{exenum}
\x
$1+i$
\x
$1-i$
\x
$1+i\sqrt3$
\x
$-2\sqrt3+2i$
\x
$-5$
\x
$e^x$, where $z=2+i\frac{pi}4$
\x
$5$
\x
$i$.
\end{exenum}

\ex{6.7.2}
Let $z_1=\sqrt3+i$ and $z_2=1-i$.
Write each of the following complex numbers
in the exponential form $|z|e^{it}$
and plot it in the complex plane.
\begin{exenum}
\x
$z_1$
\x
$z_2$
\x
$z_1z_2$
\x
$\frac{z_1}{z_2}$
\x
$2z_1$
\x
$(z_1)^6$.
\end{exenum}

\ex{6.7.3}
Find the real and imaginary parts of each
of the following complex numbers.
\begin{exenum}
\x
$e^{i\pi}$
\x
$2e^{i(\frac{\pi}4)}$
\x
$e^{i\pi}2e^{i(\frac{\pi}4)}$
\x
$e^{i\pi} + 2e^{i(\frac{\pi}4)}$
\x
$\sqrt{34}e^{it}$, where $t=\arcsin \frac{5}{\sqrt{34}}$
\x
$\sqrt{13}e^{it}$, where $\sin t= \frac{-2}{\sqrt{13}}$.
\end{exenum}

\ex{6.7.4}
If $z_1=|z_1|e^{it}$, what is the exponential
form of its complex conjugate $\conj{z_1}$?

\ex{6.7.5}
Derive \thref{6.7.3'} and \thref{6.7.4'}
using \thref{6.7.1'} and \thref{6.7.2'}.

\ex{6.7.6}
Let $n$ be a positive integer, and let $z^n$
be defined as in the text.
If $z \ne 0$, define
$z^{-n} = \frac1{z^n}$,
and then show that
$(e^z)^{-n} = e^{-nz}$.
As a result, we know that
Theorem \thref{6.7.7} holds for all integers.

\ex{6.7.7}
Find and plot the $n$th roots of $z$
in each of the following cases.
\begin{exenum}
\x
$n=3$ and $z=8i$
\x
$n=2$ and $z=i$
\x
$n=3$ and $z=2$
\x
$n=4$ and $z=1$
\x
$n=5$ and $z=2i$
\x
$n=3$ and $z=1+i\sqrt3$.
\end{exenum}

\ex{6.7.8}
How would you define the function $5^z$?

\ex{6.7.9}
\begin{exenum}
\x
Using the equation $e^{ix}=\cos x + i \sin x$
and the fact that $(e^{ix})^n = e^{inx}$,
prove that
\[
\cos nx + i \sin nx = (\cos x + i \sin x)^n
.
\]
This is known as \dt{de Moivre's Formula}.
\x
Using de Moivre's Formula and the Binomial Theorem,
find trigonometric identities for $\cos 3x$ and
$\sin 3x$ in terms of $\cos x$ and $\sin x$.
\end{exenum}

\ex{6.7.10}
Every complex-valued function $f$
of a real variable determines two real-valued
functions $f_1$ and $f_2$ of a real variable
defined by
\[
f_1(x) = \mbox{real part of $f(x)$,}
\]
\[
f_2(x) = \mbox{imaginary part of $f(x)$.}
\]
Thus $f(x) = f_1(x) + if_2(x)$ for every $x$ in
the domain of $f$.
We define the \dt{derivative} $f^\prime$
by the formula
\[
f^\prime(x) = f_1^\prime(x) + i f_2^\prime(x)
.
\]
Applying this definition to the function $f(x) = e^{ix}$,
show that
\[
\ddx e^{ix} = i e^{ix}
.
\]

\end{exercises}
