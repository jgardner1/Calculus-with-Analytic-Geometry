\begin{exercises}

\ex{11.6.1}
Find the following derivatives.
\begin{exenum}
\x
$\ddx \cosh 5x$
\x
$\ddx (\cosh^23x+\sinh^23x)$
\x
$\ddx \ln \cosh(x^2+1)$
\x
$\ddx \sinh \sqrt{x^2+1}$
\x
$\ddx \tanh x$
\x
$\ddx \sech x$
\x
$\ddx \csch x$
\x
$\ddx \coth x$
\x
$\ddt \tanh frac1{1+t^2}$
\x
$\ddx a \cosh \left(\frac xa \right)$.
\end{exenum}

\ex{11.6.2}
Find the following integrals.
\begin{exenum}
\x
$\int \sinh 7x\;dx$
\x
$\int \cosh \frac t2\;dt$
\x
$\int \sinh 3x \cosh^3 3x\;dx$
\x
$\int \tanh x\;dx$
\x
$\int \frac{\sech^2x}{\tanh x} dx$
\x
$\int 2x \sinh (2x^2+1)\;dx$
\x
$\int \tanh^52x \sech^22x\;dx$
\x
$\int \coth x \ln \sinh x\;dx$
\x
$\int \cosh^2 x\;dx$
\x
$\int \sinh^2 x\;dx$.
\end{exenum}

\ex{11.6.3}
Prove the following identities.
\begin{exenum}
\x
$\cosh 2x = \cosh^2x + \sinh^2x$
\x
$\sinh 2x = 2 \sinh x \cosh x$
\x
$1-\tanh^2x = \sech^2x$
\x
$\coth^2x - 1=\csch^2x$.
\end{exenum}

\ex{11.6.4}
Prove that $\cosh x$ is an even function and that
$\sinh x$ is an odd function.

\ex{11.6.5}
Find the general solution of each of the following
differential equations in terms of the hyperbolic
functions.
\begin{exenum}
\x
$\deriv2y = 4y$
\x
$(D^2-7)y = 0$
\x
$\deriv2y - 9y = 5e^{2x}$
\x
$(D^2-k^2)y = x + \sin x$
\x
$(D^2-16)y = 5\sinh 8x$
\x
$(D^2-16)y = 5 \cosh 4x$.
\end{exenum}

\ex{11.6.6}
Prove that $\sinh x = 0$ if and only if $x=0$.

\ex{11.6.7}
Find $\lim_{x\goesto\infty} \frac{\sinh x}{\cosh x}$.

\ex{11.6.8}
Identify and draw the curve defined parametrically by
\[
\dilemma{x(t) = \cosh t,}
{y(t) = \sinh t, \quad -\infty < t < \infty.}
\]

\ex{11.6.9}
\begin{exenum}
\x
Draw the region $R$ bounded by the $x$-axis, the
hyperbola $x^2-y^2 = 1$,
and the straight line joining the origin to the point
$(x,y)$ on the hyperbola defined by $x = \cosh t$
and $y = \sinh t$, for an arbitrary $t>0$.
\x
Compute the area of the region $R$.
\end{exenum}

\ex{11.6.10}
Find the arc length of the graph of the equation
$y=3\cosh \left(\frac x3\right)$
from the point $(0,3)$ to the point
$(6, 3\cosh 2)$.

\ex{11.6.11}
Compute the following derivatives.
\begin{exenum}
\x
$\ddx \arctanh x$
\x
$\ddx \arcsinh x$.
\end{exenum}

\ex{11.6.12}
Sketch the graph of the following equations.
\begin{exenum}
\x
$y = \tanh x$
\x
$y = \arccosh x$.
\end{exenum}


\end{exercises}
