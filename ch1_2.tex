\section{Ordered Pairs of Real Numbers, the $xy$-Plane, Functions.}
\label{sec 1.2} 

The set whose members consist of just the two elements $a$ and $b$
is denoted $\{ a, b \}$.
The notation is not perfect
because it suggests that the members $a$ and $b$ have been ordered:
$a$ is written first and $b$ second.
Actually no ordering is present because $\{ a, b \} = \{ b, a \}$.
Note also that if $a = b$,
then $\{ a, b \} = \{ b, a \} = \{ a \}$.
It can happen, however, that the ingredient of order is essential.
We therefore introduce the notion of
an \dt{ordered pair} $(a, b)$
whose first member is $a$ and whose second member  is $b$.
The characteristic property of ordered pairs is
\[
(a, b) = (c, d)
\provx{if and only if  $a = c$ and $b=d$.}
\]
In particular $(a, b) = (b, a)$ \emph{if and only if} $a = b$.
In Section \secref{1.1}
we saw that the set $\R$ of all real numbers
can be thought of as a straight line.
We shall now show that every ordered pair $(a, b)$
of real numbers $a$ and $b$
can be identified with a point in a plane.
This brings up a notational problem:
Is $(5, 7)$ the ordered pair of real numbers
or is it the open interval consisting of all $x$
such that $5 < x < 7$?
The answer is that it is impossible to tell
out of context---just as it is impossible to tell whether the word
``well" is the noun or the adverb.

Consider two distinct real number lines drawn in a plane
so that they intersect at the number $0$ on each line.
One of the lines is traditionally drawn horizontal
and called the \dt{$x$-axis},
and the other is made perpendicular to it
and called the \dt{$y$-axis}.
The orientation is chosen so that the number $1$ on the $x$-axis
lies to the right of $0$,
and the number $1$ on the $y$-axis is above $0$.
It is also customary to use the same scale of distances on both axes.
For every ordered pair $(a, b)$ of real numbers,
let $L_a$ be the line parallel to the $y$-axis
that cuts the $x$-axis at $a$,
and let $M_b$ be the line parallel to the $x$-axis
that cuts the $y$-axis at $b$.
We assign the point of intersection of $L_a$ and $M_b$
to the ordered pair $(a, b)$ (see Figure \figref{1.6}).
The numbers $a$ and $b$ are called the \dt{coordinates} of the point.
$a$ is the \dt{$x$-coordinate}
(or \dt{abscissa})
and $b$ is the \dt{$y$-coordinate}
(or \dt{ordinate}).

\putfig{3truein}{scanfig1_6}{}{fig 1.6}

If the pairs $(a, b)$ and $(c, d)$ are not equal,
then the points in the plane assigned to them will be different.
In addition, every point in the plane has a number pair assigned to it:
Starting with a point, draw the two lines through it
which are parallel to the $x$-axis and the $y$-axis.
One line cuts the $x$-axis at a number $a$,
and the other cuts the $y$-axis at $b$.
The ordered pair $(a, b)$ has the original point assigned to it.
It follows that our assignment
\[
\mbox{pair} \goesto \mbox{point}
\]
is a one-to-one correspondence
between the set of all ordered pairs of real numbers,
which we denote by $\R^2$,
and the set of all points of the plane.
It is convenient simply to identify $\R^2$ with the plane
together with the two axes.

\begin{example}
\label{exam 1.2.1}
Plot the points
$(1, 2), (-2, 3), (0, 1), (4, 0), (-2, -3)$, and $(2, -3)$
on the $xy$-plane
(see Figure \figref{1.7}).
\end{example}

\putfig{3truein}{scanfig1_7}{}{fig 1.7}

The usefulness of the idea of an ordered pair
is by no means limited to pairs of real numbers.
In plane geometry, for example,
we may consider the set of all ordered pairs $(T, p)$
in which $T$ is a triangle and $p$ is the point of intersection
of its medians.
In the three-dimensional extension of the $xy$-plane,
the set $\R^3$ of all ordered triples $(a, b, c)$ of real numbers
is identified with the set of all points in three-dimensional space.
The definition of an ordered triple
can be reduced to that of an ordered pair
by defining $(a, b, c)$ to be $((a, b), c)$.

\putfig{3truein}{scanfig1_8}{}{fig 1.8}

Let $P = (a, b)$ and $Q = (c, d)$
be arbitrary elements in the set
$\R^2$ of all ordered pairs of real numbers.
We define the \dt{distance} between $P$ and $Q$ by the formula
\begin{equation}
\label{eq1.2.1}
\dist(P, Q) = \sqrt{(a - c)^2 + (b - d)^2}.     
\end{equation}
Three simple corollaries of this definition are:

\begin{prop}
\label{thm 1.2.1}
$\dist(P, Q) \geq 0$; i.e., distance is never negative.
\end{prop}
\begin{prop} %(2.2)
\label{thm 1.2.2}
$\dist(P, Q) = 0$
if and only if $P = Q$. 
\end{prop}
\begin{prop} %(2.3)
\label{thm 1.2.3}
$\dist(P, Q) = \dist (Q, P)$. 
\end{prop}

Another consequence of (1)
is that it is no longer simply a matter of tradition and convenience
that we draw the $y$-axis perpendicular to the $x$-axis.
It follows from consideration of the Pythagorean Theorem
and its converse (see Figure \figref{1.8})
that the above definition of distance between elements of $\R^2$
corresponds with our geometric notion of the distance between points
in the Euclidean plane
if and only if the two coordinate axes are perpendicular
and the scales are the same on both.

\putfig{3truein}{scanfig1_9}{}{fig 1.9}

\begin{example}
\label{exam 1.2.2}
Let $C$ be the subset of the $xy$-plane
consisting of all points whose distance from (1,1) is equal to 2.
Thus $C$ is the circle shown in Figure \figref{1.9}.
If $(x, y)$ is an arbitrary point in the $xy$-plane,
its distance from $(1,1)$
is equal to
$\sqrt{(x - 1)^2 + (y - 1)^2}$.
Hence, $(x, y)$ belongs to $C$
if and only if
\begin{equation}
\sqrt{(x - 1)^2 + (y - 1)^2} = 2.  
\label{eq1.2.2}
\end{equation}
Numbers $x$ and $y$ satisfy (2)
if and only if they satisfy 
\begin{equation}
(x - 1)^2 + (y - 1)^2 = 4.        
\label{eq1.2.3}
\end{equation}
Thus $C$ is the set of all ordered pairs $(x, y)$
that satisfy (3)---or that satisfy (2).
Either (2) or (3) is therefore called
\dt{an equation of the circle} $C$.
\end{example}

The set of all points $(x, y)$ in the plane
that satisfy a given equation is called the
\dt{graph} of the equation.
Hence, in the above example,
the circle $C$ is the graph of the equation
$(x - 1)^2 + (y - 1)^2 = 4$.

\begin{example}
\label{exam 1.2.3}
Let $L$ be the set of all ordered pairs $(x, y)$
such that $y = 2x - 3$.
For each real number $x$,
there is one and only one number $y$
such that $(x, y)$ belongs to $L: y = 2x - 3$.
To see what $L$ looks like,
we plot five of its points (see Table \tabref{1.1}).
As shown in Figure \figref{1.10},
all these points lie on a straight line.
In Section \secref{1.5}
we shall justify the natural conjecture
that this straight line is the set $L$.
\end{example}

\begin{table}
\centering
\begin{tabular}{r|c} \hline
$x$  &  $y=2x - 3$     \\ \hline
 -1  &   -5            \\
  0  &   -3            \\
  1  &   -1            \\
  2  &    1            \\
  3  &    3            \\ \hline
\end{tabular}
\caption{}
\label{table 1.1}
\end{table}

\putfig{2.5truein}{scanfig1_10}{}{fig 1.10}

\begin{example}
\label{exam 1.2.4}
The set of all pairs $(x, y)$
such that $y^2 = x$ is the curve shown in Figure \figref{1.11}.
This curve is a parabola,
one of the conic sections,
which are studied in greater detail in Chapter \chref{3}.
At present we shall be satisfied with plotting a few points and connecting
them with a smooth curve (see Table \tabref{1.2}).

\putfig{3truein}{scanfig1_11}{}{fig 1.11}

\begin{table}
\centering
\begin{tabular}{r|r} \hline
$y$        &  $x$      \\ \hline
  0          &   0         \\
$\pm1$  &   1         \\
$\pm2$  &   4         \\ \hline
\end{tabular}
\caption{}
\label{table 1.2}
\end{table}
\end{example}

A \dt{function} $f$ is any set $f$ of ordered pairs
such that whenever $(a, b)$ and $(a, c)$ belong to $f$,
then $b = c$.
Note that every subset of the $xy$-plane is a set of ordered pairs,
but not every subset is a function.
In particular, the parabola in Example \exampref{1.2.4} is not,
because it contains both (4, 2) and $(4, -2)$.
On the other hand,
the straight line in Example \exampref{1.2.3} is a function.
This condition that a function
must never contain two pairs $(a, b)$ and $(a, c)$ with $b \neq c$
means geometrically that a subset of the $xy$-plane is a function
if and only if it never intersects a line parallel to the $y$-axis
in more than one point.
Hence it is an easy matter to decide which of the following sets
are functions and which are not:
\begin{quote}
(i) The set $f$ of all pairs $(x, y)$ such that $y = x + 1$.

(ii) The set $g$ of all pairs $(x, y)$ such that $x^2 + y^2 = 1$.

(iii) The set $F$ of all pairs $(x, y)$ such that $y = x^2 + 2x + 2$.

(iv) The set $h$ of all pairs $(x, y)$ such that $2x + 3y = 1$.

(v) The set $G$ of all pairs $(x, y)$ such that $y = \sqrt {x + 2}$.

(vi) The set $H$ of all pairs $(x, y)$ such that $y^4 = x$.
\end{quote}
The sets $f$, $F$, $h$, and $G$ are functions,
but $g$ and $H$ are not.

The \dt{domain} of a function $f$ is the set of all elements $a$
for which there is a corresponding $b$
such that $(a, b)$ belongs to $f$.
Analogously, the \dt{range} of $f$ is the set of all elements $b$
for which there is an $a$
such that $(a, b)$ belongs to $f$.
In (i),
the domain of $f$ is the set $\R$ of all real numbers and so is the range.
On the other hand, in (iii),
although the domain of $F$ is equal to $\R$,
the range is the interval consisting of all real numbers $y \geq 1$,
because we can write $x^2 + 2x + 2 = (x + 1)^2 + 1 \geq 1$.

If a pair $(a, b)$ belongs to a function $f$,
we call $b$ the \dt{value of $f$ at $a$} and write $b = f(a)$.
Note that the meaning of $f(a)$ is unambiguous
only because the definition of a function
forbids having $(a, b)$ and $(a, c)$ both belong to $f$ if $b \neq c$.
Therefore the second member of any ordered pair that belongs to $f$
is determined by the first member.

\begin{example}
\label{exam 1.2.5}
In (i),
\[
\begin{array}{ll}
f(x) = x + 1,                &f(a) = a + 1, \vspace{.1in}\\
f(0) = 1,                      &f(3 + 4) = (3 + 4) + 1 = 8, \vspace{.1in}\\
f(-1) = -1 + 1 = 0,       &f(a + b) = a + b + 1 .
\end{array}
\]
In (v),
\[
\begin{array}{ll}
G(x) = \sqrt{x + 2},       &G(2x + y) = \sqrt{2x + y + 2}, \vspace{.1in}\\
G(0) = \sqrt2,               &G(-2) = 0, \vspace{.1in}\\
G(2) = 2,                      &G(-3) \mbox{is not defined} .
\end{array}
\]
\end{example}

To each element $a$ in the domain of a function $f$
there corresponds a value $f(a)$ in the range.
This correspondence between domain and range,
which is pictured in Figure \figref{1.12},
\putfig{3truein}{scanfig1_12}{}{fig 1.12}
is the central idea in the definition of a function.
Thus the function $f$ that consists of all ordered pairs $(x, y)$
such that $y = x^2$ and $-1 \leq x \leq 2$
is interpreted as the rule of correspondence
which assigns to each number in the interval $[-1, 2]$ its square.
We can describe $f$ completely and simply by writing
\[
f(x) = x^2,       \;\;\;     -1 \leq x \leq 2.
\]
Examples of other functions are
\[
g(x) = \sqrt{x - 1}, \prov{-1 \leq x < \infty},
\]
\[
F(x) = x^2, \prov{-\infty < x < \infty,}
\]
\[
h(x) = \frac{x}{x + 2}, \prov{x \neq -2} .
\]
Note that the functions $f$ and $F$ immediately above
are \emph{not} equal,
although $f$ is a subset of $F$.
Two functions are equal
if they are one and the same set of ordered pairs.
It follows that

\begin{prop}
\label{thm 1.2.4}
Functions $f$ and $g$ are equal
if and only if they have the same domain $D$
and $f(x) = g(x)$ for every element $x$ in $D$. 
\end{prop}

Thus any complete description of a function
must include a description of its domain.
Sometimes this information is in fact omitted.
We shall adopt the convention that
if no explicit description of the domain of a function is given,
then its domain is assumed to be the largest set of real numbers
that makes sense.
For example, the domain of the function $H$ defined by
\[
H(x) = \frac{1}{x^2 - x - 2} = \frac{1}{(x + 1)(x - 2)}
\]
is assumed to be the entire set of real numbers
with the exception of $-1$ and 2.

It is sometimes helpful to think of a function
as a computing machine.
Imagine a computing machine,
named $f$,
which is provided with an input tape,
an output tape,
and a button (see Figure \figref{1.13}).
\putfig{3truein}{scanfig1_13}{A computing machine.}{fig 1.13}
One writes a number $x$ on the input tape
and pushes the button.
If $x$ is one of the inputs which the machine will accept,
i.e., if $x$ is in the domain of $f$,
the machine whirs contentedly and prints an output,
which we denote $f(x)$,
on the output tape.
If $x$ is not in the prescribed domain,
either nothing happens or a red light flashes.

We have already seen that
one of the best ways of describing a subset of $\R^2$
is to draw a picture of it.
If this subset happens to be a function,
we call the picture the graph of the function.
More specifically,
if a function $f$ is a subset of $\R^2$,
its \dt{graph} is the set of all points in the plane
that correspond to ordered pairs of the form $(x, f(x))$.
Note that the graph of $f$
depends on the correspondence between ordered pairs and points;
i.e., it depends on the choice of axes.
To illustrate this,
in Figure \figref{1.14}
\putfig{2truein}{scanfig1_14}{Two graphs of the function $f(x)=x^3$.}{fig 1.14}
we have drawn the graph of the function $f$
defined by $f(x) = x^3$
for two sets of axes.
For a single choice of axes,
we simply identify ordered pairs and points,
and under this identification
a function and its graph become the same thing.

Most of the functions encountered in an introduction to calculus
are defined by means of a single equation;
e.g., $h(x) = x^3 + 3$.
It is a bad mistake, however,
to assume that this is always true.
The function $F$ given by
\[
F(x) = \left \{ \begin{array}{ll}
        x^2 + 1           &\mbox{if $x \geq 0$},\\
        -\frac{x}{2}      &\mbox{if $x < 0$},
                   \end{array}
\right. 
\]
requires two equations for its definition.
The graph of $F$ is shown in Figure \figref{1.15}.
\putfig{2.5truein}{scanfig1_15}
{A function not defined by a simple formula.}{fig 1.15}
Another function,
which is so wild that it is impossible to draw its graph,
is the following:
\[
g(x) = \left \{ \begin{array}{ll}
				0        & \mbox{if $x$ is rational,}\\
				1        & \mbox{if $x$ is irrational.}							\end{array}
\right.
\]
The ordered pairs that comprise a function
are not necessarily pairs of numbers.
An example is the function,
mentioned earlier in this section,
which assigns to each triangle
the point of intersection of its medians.
It is possible for the domain of a function to be
a set of ordered pairs.
Consider the function $f$ consisting of all ordered pairs $((x, y), z)$,
where $x, y$, and $z$ are numbers that satisfy
$x \geq y$ and $z = 2x^2 + y^2$.
We describe this function simply as follows:
\begin{equation}
f(x, y) = 2x^2 + y^2,          \;\;\;  x \geq y.  
\label{eq1.2.4}
\end{equation}
As a final example of a function,
consider the rule of correspondence that assigns to each person
his or her male parent.

As we have indicated,
the definition of a function is appallingly general.
One of our tasks is to delineate properly
the kinds of functions studied in calculus.
To begin with,
a function $f$ is said to be \dt{real-valued}
if its range is a subset of $\R$, the set of real numbers.
If the domain of $f$ is a subset of $\R$,
we call $f$ a \dt{function of a real variable}.
The function $f(x, y)$ defined in \eqref{1.2.4}
has as its domain a subset of $\R^2$.
It is a real-valued function of two real variables.
For the most part,
a first course in calculus is a study of
real-valued functions of one real variable.

