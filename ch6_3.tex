\section{Other Trigonometric Functions.}
The other trigonometric functions are the tangent, cotangent, secant, and cosecant. They are abbreviated $\tan$, $\cot$ (or ctn), $\sec$, and $\csc$, respectively, and the definitions are

$$
\begin{array}{ll}
\tan x = \frac{\sin x}{\cos x}     &    \sec x = \frac{1}{\cos x}\\
\\
\cot x = \frac{\cos x}{\sin x}     &    \csc x = \frac{1}{\sin x}.
\end{array}
$$
\smallskip

Unlike $\sin$ and $\cos$, these functions are not defined for all real values of $x$ since the
denominators in the defining expressions are zero for some values of $x$. The set of all solutions
to the equation $\cos x = 0$ is the set consisting of all odd multiples of $\frac{\pi}{2}$ . 
Hence $\tan x$ and $\sec x$ are defined if and only if $x$ is not an odd multiple of $\frac{\pi}{2}$. Similarly, $\cot x$ and $\csc x$ are defined for all real numbers $x$ except integer multiples of $\pi$.

Although sine and cosine were first defined with a domain of real numbers, we have shown that
they also can be considered as functions with a domain of angles. Since the other four functions
are defined in terms of sine and cosine, they may also be regarded as functions with a domain of
angles. Thus it makes sense to speak of the tangent of the angle $\alpha$, written $\tan \alpha$, 
and of the cosecant of an angle of $30^{\circ}$, written $\csc 30^{\circ}$. The former is defined if 
and only if the radian measure of $\alpha$ is not an odd multiple of $\frac{\pi}{2}$ or, alternatively, 
if the degree measure of $\alpha$ is not an odd multiple of 90. The latter is the reciprocal of 
$\sin 30^{\circ}$ and is equal to $\frac{1}{\frac{1}{2}} = 2$.

Two useful trigonometric identities, which are simply alternative statements of the basic equation $\cos^{2}x + \sin^{2}x = 1$, are derived as follows:
Dividing first by $\cos^{2}x$, we have 

$$
\begin{array}{ccccc}
\frac{\cos^{2}x}{\cos^{2} x} &+& \frac{ \sin^{2} x }{\cos ^{2} x} &=& \frac{1}{\cos^{2}{x}}\\ 
\\
          ||                               &  &                     ||                         & &                    ||\\
\\
          1                               &  &               \tan^{2} x                  & &             \sec^{2}x, 
\end{array}
$$
\noindent hence $1 + \tan^{2}x = \sec^{2}x$. On the other hand, if we divide by $\sin^{2}x$, we have


$$
\begin{array}{ccccc}
\frac{\cos^{2}x}{\sin^{2} x} &+& \frac{ \sin^{2} x }{\sin ^{2} x} &=& \frac{1}{\sin^{2}{x}}\\ 
\\
            ||                            &  &              ||                              &  &                ||\\
\\
         \cot^{2} x                  &  &              1                              &  &           \csc^{2}x, 
\end{array}
$$
%302 TRIGONOMETRIC FUNCTIONS [CHAP. 6 
\noindent and so $\cot^{2}x + 1 = \csc^{2}x$. Summarizing, we write

\begin{theorem} %( 3.1 ) 
\begin{eqnarray*}
1 +\tan^{2}x &=& \sec^{2}x, \\
\cot^{2}x + 1 &=& \csc^{2}x.
\end{eqnarray*}
\end{theorem}

Another formula which we shall find useful is that for the tangent of the difference of two numbers, $a - b$, in terms of $\tan a$ and $\tan b$. First,
$$
\tan (a - b) = \frac{\sin (a - b)}{\cos(a - b)} = \frac{\sin a \cos b - \cos a \sin b}{\cos a \cos b + \sin a \sin b}.
$$

Dividing both numerator and denominator by $\cos a \cos b$, we get


\begin{eqnarray*}
\tan (a - b) &=& \frac{\frac{\sin a \cos b}{\cos a \cos b} - \frac{\cos a \sin b}{\cos a \cos b}}{\frac{\cos a \cos b}{\cos a \cos b} + \frac{\sin a \sin b}{\cos a \cos b}}\\
&=& \frac{\frac{\sin a}{\cos a} - \frac{\sin b}{\cos b}}{ 1 + \frac{\sin a}{\cos a} \frac{\sin b}{\cos b}} .
\end{eqnarray*}

\noindent Hence

\begin{theorem} %( 3.2 ) 
$$
\tan(a-b) = \frac{\tan a - \tan b}{1 + \tan a \tan b}.
$$
\end{theorem}

The trigonometric identities developed in this section are handy tools, and we shall not hesitate
to use them. In themselves, however, they are of secondary importance. Any one of them can be
derived quickly and in a completely routine way from the basic identities in $\sin$ and $\cos$ derived
in Section 1.

An important application of the tangent function is in connection with the slope of a straight line. We define the angle of inclination $\alpha$ of a straight line $L$ as follows: If $L$ is horizontal, then $\alpha = 0$. If $L$ is not horizontal, then it intersects an arbitrary horizontal line $H$ in a single point $P$. Let $\alpha$ be the angle with vertex $P$, initial side the part of $H$ to the right of $P$, terminal side the part of $L$ above $P$, and whose measure in radians satisfies the inequality $0 < \alpha < \pi$ (see Figure 10). We contend that

\begin{theorem} %(3.3) 
The slope of a line is equal to the tangent of its angle of inclination.
\end{theorem}


\begin{proof}
We refer again to Figure 10. The given line is $L$, its inclination is $\alpha$, and its slope is $m$. If $L'$ is drawn through the origin parallel to $L$, then $L'$ also has
inclination $\alpha$ and slope $m$. Furthermore, the point $(\cos \alpha, \sin \alpha)$ lies on $L'$.
Since (0,0) also lies on $L'$, the definition of slope yields
$$
m = \frac{\sin \alpha - 0}{\cos \alpha - 0} = \frac{\sin \alpha}{\cos \alpha} = \tan \alpha,  
$$
which completes the proof. If $L$ is vertical, its slope is not defined. But then the angle of inclination is $\frac{\pi}{2}$ and $\tan\frac{\pi}{2}$ is not defined either.
\end{proof}


\putfig{4truein}{scanfig6_10}{}{fig 6.10}
\putfig{2.75truein}{scanfig6_11}{}{fig 6.11}

The importance of (3.2) is apparent when we try to determine the angle between two nonvertical intersecting lines. Let $\alpha$ be the angle of inclination of one line and, $\beta$ the angle of inclination of the other. For convenience, we assume that $\alpha > \beta$.  It follows that $0 < \alpha - \beta < \pi$ and, from Figure 11
%304 rRlCONOMETRlC FUNCTIONS [C~AP 6
that the difference $\alpha - \beta$ is an angle between the two lines. We denote the slope of the first line by $m_{1}$, and that of the second by $m_{2}$. That is, we have $m_{1} = \tan \alpha$ and $m_{2} = \tan \beta$. By (3.2),

$$
\tan (\alpha - \beta) = \frac{\tan \alpha - \tan \beta}{1 + \tan \alpha \tan\beta} = \frac{m_{1} - m_{2}}{1 + m_{1}m_{2}}.
$$
\noindent If this number is positive, then $\alpha - \beta$ is the acute angle between the lines. If this number is negative, then $\alpha - \beta$ is the obtuse angle between the lines. If this number is unclefined, then $\alpha - \beta = \frac{\pi}{2}$ and the lines are perpendicular. The number $\frac{m_{1} - m_{2}}{1 + m_{1}m_{2}}$ is unclefined if and only if $1 + m_{1}m_{2} = 0$. Since this equation is equivalent to $m_{1} m_{2} = - 1$, it follows that we have proved the statement made on page 44 that two nonvertical lines with slopes $m_{1}$ and $m_{2}$, are perpendicular if and only if $m_{1}m_{2} = - 1$.

The formulas for the derivatives of the remaining four trigonometric functions are found using
the derivatives of $\sin$ and $\cos$ together with the usual rules of differentiation. They are

 
\begin{theorem} %( 3.3 ) 
\begin{eqnarray*}
\frac{d}{dx} \tan x &=& \sec^{2}x,\\
\frac{d}{dx} \cot x &=& - \csc^{2}x,\\
\frac{d}{dx} \sec x &=& \sec x \tan x, \\
\frac{d}{dx} \csc x &=&  - \csc x \cot x.
\end{eqnarray*}
\end{theorem}

Proving the first of these, we have

\begin{eqnarray*}
\frac{d}{dx} \tan x &=& \frac{d}{dx} \frac{\sin x}{\cos x} = \frac{\cos x \frac{d}{dx} \sin x - \sin x \frac{d}{dx} \cos x}{\cos^{2} x}   \\
&=& \frac{\cos^{2} x + \sin^{2} x}{\cos ^{2} x} = \frac{1}{\cos^{2} x} = \sec^{2} x.
\end{eqnarray*}

\noindent The others are left as exercises. We observe the following mnemonic device. 
From any one of the six formulas for differentiating trigonometric functions another one is obtained by adding the prefix ``co" to every function which does not have one, removing the prefix ``co" from each function which has it already, and changing the sign. For example, this procedure transforms the equation $\frac{d}{dx} \sec x = \sec x \tan x$ into $\frac{d}{dx} \csc x= -\csc x \cot x$ and transforms
%SEC. 3] OTHER TRIGONOMETRIC FUNCTIONS 305
$\frac{d}{dx} \cos x = - \sin x$ into $\frac{d}{dx} \sin x= \cos x$. Hence the number of derivative formulas which need to be memorized can be cut in half.

The integrals corresponding to the above derivatives are

\begin{theorem} %(3.4)
\begin{eqnarray*}
&\int&\sec^{2}x dx = \tan x + c,\\
&\int&\csc^{2}x dx = -\cot x + c,\\
&\int&\sec x \tan x dx = \sec x + c,\\ 
&\int&\csc x\cot x dx = - \csc x + c.
\end{eqnarray*}
\end{theorem}

%EXAMPLE 1. 
\begin{example} Find the following integrals:


\begin{quote}
\begin{description}
\item[(a)] $\int x^{2} \sec^{2} (x^{3} + 1 ) dx,$
\item[(b)] $\int \tan x dx,$
\item[(c)] $\int \csc^{2}x \cot^{5}x dx.$
\end{description}
\end{quote}
\noindent In (a) we observe that $\frac{d}{dx}(x^{3} + 1) = 3x^{2}$, or, equivalently,
$$
x^2 = \frac{1}{3} \frac{d}{dx} (x^{3} + 1).
$$
\noindent Hence


\begin{eqnarray*}
\int x^{2} \sec^{2}(x^{3} + 1) dx &=& \frac{1}{3} \int [\sec^{2} (x^{3} + 1)] \frac{d}{dx} (x^{3} + 1)dx\\
&=& \frac{1}{3} \tan (x^{3}+ 1) + c.
\end{eqnarray*}

\noindent For (b), we have $\int \tan x dx = \int \frac{\sin x}{\cos x} dx$. If $u = \cos x$, then 
$\frac{du}{dx} = -\sin x$, and so


\begin{eqnarray*}
\int \tan x dx &=& \int  \frac{\sin x}{\cos x} dx = - \int \frac{1}{u} \frac{du}{dx} dx\\
                     &=& - \ln |u| + c = - \ln |\cos x| + c.
\end{eqnarray*}

%306 TRlGONOhIETRlC FUNCTIONS [CHAP. 6
\noindent Finally, to do (c), we see that since $\frac{d}{dx} \cot x = - \csc^{2}x$, the integral is, 
except for a minus sign, of the form $\int u^{5} \frac{du}{dx} dx$. Thus


\begin{eqnarray*}
\int \csc^{2} x \cot^{5}x dx &=& - \int (\cot^{5}x) \frac{d}{dx} \cot x dx\\
&=& - \frac{1}{6} \cot^{6}x + c.
\end{eqnarray*}

\noindent Each of these integrals can be checked by differentiation.
\end{example}

The graph of $\tan x$ is an interesting curve, which we now describe. Note, first of all, that $\tan$ is an odd function, 

$$
\tan(-x) = \frac{\sin(-x)}{\cos(-x)} = \frac{-\sin x}{\cos x} = - \tan x,
$$
\noindent and the graph is therefore symmetric about the origin. Moreover, 

\begin{eqnarray*}
\tan(x + \pi) &=& \frac{\sin(x + \pi) }{\cos ( x + \pi)} = \frac{\sin x \cos \pi + \cos x \sin \pi}
{\cos x \cos \pi - \sin x \sin \pi} \\
&=& \frac{-\sin x}{-\cos x} = \tan x.
\end{eqnarray*}

\noindent Thus $\tan$ is a periodic function with period $\pi$. The slope of the graph is given by the derivative,

$$
\frac{d}{dx}\tan x = \sec^{2}x,
$$
\noindent which is positive for every value of $x$ for which $\tan x$ is defined. Hence $\tan x$ is a strictly increasing function in the interval $-\frac{\pi}{2} < x < \frac{\pi}{2}$. From the definition, $\tan x = \frac{\sin x}{\cos x}$, we see that $\tan x$ is positive when both functions $\cos x$ are positive, as they are for $0 < x < \frac{\pi}{2}$; is zero when $\sin x = 0$, as it is for $x = 0$; and is negative when the two functions have opposite sign, as they do for $-\frac{\pi}{2} < x < 0$. We also see that $\tan x $ takes on arbitrarily large positive values as $x$ approaches $\frac{\pi}{2}$ from the left, since $\sin x$ approaches 1 and $\cos x$ approaches 0. Thus 

$$
\lim_{x \rightarrow \frac{\pi}{2}-} \tan x = \infty.
$$
% sec. 3] OTHER TRIGONOMETRIC FUNCTIONS ~ 307
\noindent The second derivative is given by 

$$
\frac{d^2}{dx^2} \tan x = 2 \sec^{2} x \tan x = \left\{
\begin{array}{ll}
<0    & \;\;\;\mbox{if}\; - \frac{\pi}{2} < x < 0,\\
 = 0  & \;\;\;\mbox{if}\;  x = 0, \\
 > 0  & \;\;\;\mbox{if}\;   0 < x < \frac{\pi}{2},
\end{array}
 \right .
$$
\noindent from which it follows that the graph is concave downward for $-\frac{\pi}{2} < x < 0$, 
concave upward for $0 < x < \frac{\pi}{2}$ and has a point of inflection at the origin.
Combining all these facts with the few isolated values shown in Table 3, we obtain the graph
shown in Figure 12.

%Figure 12 
\putfig{4.5truein}{scanfig6_12}{}{fig 6.12}
\medskip

%TABLE 3
\begin{table}
\centering
\begin{tabular}{l|l|l}\hline
$x$ & $y$ = $\tan x$ & $\frac{dy}{dx} = \sec^{2}x$ \\\hline
0 &       0        & 1\\
$\frac{\pi}{6}$ & $\frac{1}{\sqrt 3}$ = 0.58 (approx.) & $\frac{4}{3}$ = 1.33 (approx.)\\
$\frac{\pi}{4}$ & 1 & 2\\
$\frac{\pi}{3}$ & $\sqrt 3$ = 1.73 (approx.) & 4\\ \hline
\end{tabular}
\caption{}
\label{table 6.3}
\end{table}
\medskip
% 308 TRIGONOMETRIC FUNCTIONS [CHAP. 6

The graph of $\cot x$ can be obtained in the same way in which we worked out the graph of $\tan x$. However, there is a quicker way based on an identity. Since 

\begin{eqnarray*}
\tan \Bigl(x + \frac{\pi}{2} \Bigr) &=& \frac{\sin \Bigl(x + \frac{\pi}{2} \Bigr)}{ \cos \Bigl(x + \frac{\pi}{2} \Bigr)} 
= \frac{\sin x \cos \frac{\pi}{2} + \cos x \sin \frac{\pi}{2}}{ \cos x \cos\frac{\pi}{2} 
- \sin x \sin \frac{\pi}{2}}  \\
&=& \frac{\cos x}{-\sin x} = - \cot x ,
\end{eqnarray*}

\noindent we know that
$$
\cot x = - \tan \Bigl(x + \frac{\pi}{2}\Bigr) .
$$
\noindent The geometric significance of this identity is that the graph of $\cot x$ is obtained by translating (sliding) the graph of $\tan x$ to the left a distance $\frac{\pi}{2}$ and then reflecting about the $x$-axis.

From the definition, $\sec x = \frac{1}{\cos x}$, it is apparent that
$$
\sec n \pi = \frac{1}{\cos n\pi} = \{ \begin{array}{rl}
1  &\;\;\;  \mbox{if $n$ is even}, \\
-1 & \;\;\; \mbox{if $n$ is odd}.
\end{array}
$$
\noindent If $a$ is any odd multiple of $\frac{\pi}{2}$, then $\cos a = 0$, and so 

$$
\lim_{x \rightarrow a} |\sec x| = \lim_{x \rightarrow a} \frac{1}{|\cos x|} = \infty.  
$$

%Figure 13
\putfig{4.5truein}{scanfig6_13}{}{fig 6.13}
%SEC. 3] OTHER TRIGONOMETRIC FVNCTIONS  309
\noindent Moreover, on an interval where one function is increasing, its reciprocal function is decreasing, and vice versa. It follows that the over-all shape of the graph of $\sec x$ can be ascertained quite easily from the graph of its reciprocal function $\cos x$. The graph of $\sec x$ is shown in Figure 13.

The graph of $\csc x$ is related to that of $\sec x$ in the same way as the graph of $\sin x$ is related to the graph of $\cos x$.

