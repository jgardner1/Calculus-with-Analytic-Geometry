\section{Absolute and Conditional Convergence.} 
An infinite series $\sum_{i=m}^{\infty} a_i$ is said to be \textbf{absolutely convergent} if the corresponding series of absolute values $\sum_{i=m}^{\infty} |a_i|$ is convergent. If a series $\sum_{i=m}^{\infty} a_i$ converges, but $\sum_{i=m}^{\infty} |a_i|$ does not, then we say that $\sum_{i=m}^{\infty} a_i$ is \textbf{conditionally convergent.} An example of a conditionally convergent series is the alternating harmonic series: We have shown that
$$
\sum_{i=1}^{\infty} a_i = \sum_{i=1}^{\infty} (- 1)^{i+1} \frac{1}{i} = 1 - \frac{1}{2} + \frac{1}{3}  - \frac{1}{4} + \cdots  
$$
%SEC. 5] ABSOLUTE AND CONDITIONAL CONVERGENCE 503
\noindent converges, but that
$$
\sum_{i=1}^{\infty} |a_i| = \sum_{i=1}^{\infty} \frac{1}{i} = 1 + \frac{1}{2} + \frac{1}{3} + \frac{1}{4} + \cdots
$$
\noindent diverges.

There are many examples of series for which both $\sum_{i=m}^{\infty} a_i$ and $\sum_{i=m}^{\infty} |a_i|$ converge, and also many where both diverge. (In particular, for nonnegative series, the two are the same.) There is the remaining possibility that $\sum_{i=m}^{\infty} |a_i|$ might converge, and $\sum_{i=m}^{\infty} |a_i|$ diverge. However, the following theorem shows that this cannot happen.

%(5.1)  
\begin{theorem} If the infinite series $\sum_{i=m}^{\infty} a_i$ is absolutely convergent, then it is convergent.
\end{theorem}

\begin{proof}
Since $|a_i| \geq -a_i$, we have $a_i + |a_i| \geq 0$, for every integer $i \geq m$. Hence the series $\sum_{i=m}^{\infty}  (a_i + |a_i|)$ is nonnegative. Since $a_i \leq |a_i|$, we also have
\begin{equation}
a_i + |a_i| \leq |a_i| + |a_i| = 2|a_i|,  
\label{eq9.5.1}
\end{equation}
for every integer $i \geq m$. The assumption that $\sum_{i=m}^{\infty} a_i$ is absolutely convergent means that the series $\sum_{i=m}^{\infty} |a_i|$ converges, and, hence, so does the series $\sum_{i=m}^{\infty} 2|a_i|$. It therefore follows from (1) by the Comparison Test that the nonnegative series $\sum_{i=m}^{\infty}  (a_i + |a_i|)$ is convergent. We conclude from Theorem (2.2), page 485, that
$$
\sum_{i=m}^{\infty} a_i = \sum_{i=m}^{\infty} (a_i + |a_i |) - \sum_{i=m}^{\infty} |a_i| 
$$
and that $\sum_{i=m}^{\infty} a_i$ converges. This completes the proof.
\end{proof}

Thus the only possibilities for a given series are those illustrated. by the following scheme:


\begin{centering}
\begin{picture}(160,80)(0,0)
\put(0,30){\line(3,1){30}}

\put(0,30){\line(3,-1){30}}
\put(40,40){convergent}

\put(40,15){divergent}
\put(100,50){\line(3,1){30}}

\put(100,50){\line(3,-1){30}}
\put(140,35){conditionally convergent}

\put(140,60){absolutely convergent}
\end{picture}
\end{centering}
\medskip 

% EXAMPLE 1.
\begin{example} Classify each of the following infinite series as absolutely convergent, conditionally convergent, or divergent.
$$
\mbox{(a)}\;\;\; \sum_{k=1}^\infty (-1)^k \frac{1}{\sqrt {k+1}} , \;\;\; 
\mbox{(b)}\;\;\; \sum_{k=1}^\infty (-1)^k \frac{1}{2k^2 - 15} .
$$
%504 INFINITE SERIES [CEIAP. 9

If we let $a_k = (-1)^k \frac{1}{\sqrt {k+1}}$, the alternating series in (a) will converge if:
 
\begin{quote}
\begin{description}
\item[(i)] $|a_{k+1}| \leq |a_k|, \mathrm{for every integer} k \geq 1, \;\mathrm{and}$ 
\item[(ii)] $\lim_{k \rightarrow \infty} |a_k| = 0.$
\end{description}
\end{quote}

\noindent  [See Theorem (4.1), page 498.] We have
$$
|a_k| = \frac{1}{\sqrt{k + 1}}\;\;\;  \mbox{and}\;\;\;    |a_{k+1}| = \frac{1}{\sqrt{k + 2}} .
$$
\noindent Hence condition (i) becomes
$$
\frac{1}{\sqrt{k + 2}} \leq \frac{1}{\sqrt{k + 1}}, \;\;\;\mbox{for every  integer}\; k \geq 1,
$$
\noindent which is certainly true. Condition (ii) is also satisfied, since
$$
 \lim_{k \rightarrow \infty} \frac{1}{\sqrt{k + 1}} = 0,
$$
\noindent and it follows that the series $\sum_{k=1}^\infty a_k$ converges. However, it is easy to show that $\sum_{k=1}^\infty |a_k|$ diverges by either the Comparison Test or the Integral Test. Using the latter, we consider the function $f$ defined by $f(x) = \frac{1}{\sqrt{x+1}}$, which is nonnegative and decreasing on the interval $[1, \infty)$. We have $f(k) = \frac{1}{\sqrt{k+1}} = |a_k|$ and
 
\begin{eqnarray*}
\int_1^\infty f(x) dx &=& \int_1^\infty \frac{1}{\sqrt{x + 1}} dx = \lim_{b \rightarrow \infty} [2 \sqrt{x + 1}|_1^b ] \\
&=& \lim_{b \rightarrow \infty} [2\sqrt{b + 1} - 2 \sqrt2] = \infty .
\end{eqnarray*}

\noindent The divergence of the integral implies the divergence of the corresponding series $\sum_{k=1}^\infty |a_k|$, and we conclude that the series (a) is conditionally convergent.

For the series in (b), we might apply the same technique: Test first for convergence and then for absolute convergence. However, if we suspect that the series is absolutely convergent, we may save a step by first testing for absolute convergence. In this particular case, the corresponding series of absolute values is $ \sum_{k = 1}^\infty \frac{1}{|2k^2 - 15|}$. The latter can be shown to be convergent
%SEC. 5] ABSOLUTE AND CONDITIONAL CONVERGENCE  505
by the CoMparison Test. For a test series we choose the convergent series $\sum_{k=1}^\infty \frac{2}{k^2}$. The condition of the test is that the inequality

$$
\frac{1}{|2k^2 - 15|} \leq \frac{2}{k^2}
$$
\noindent must be true eventually. We shall consider only integers $k \geq 3$, since, for these values, $2k^2 \geq18$ and hence $|2k^2 - 15| = 2k^2 - 15$. For those integers for which $k \geq 3$, the inequality

$$
\frac{1}{2k^2 - 15} \leq \frac{2}{k^2}
$$
\noindent is equivalent to $k^2 \leq 4k^2 - 30$, which in turn is equivalent to $k^2 \geq 10$. The last is true for every integer $k \geq 4$. Hence

$$
\frac{1}{|2k^2 - 15|} \leq \frac{2}{k^2}, \;\;\;\mbox{for every integer}\; k \geq 4.
$$

\noindent lt follows that $\sum_{k=1}^\infty \frac{1}{|2k^2 - 15|}$ converges, and therefore that the series (b) is absolutely convergent.
\end{example}

%(5.2) 
\begin{theorem} RATIO TEST. Let $\sum_{i=m}^\infty a_i$ be an infinite series for which \linebreak
$\lim_{n \rightarrow \infty} \frac{|a_{n+1}|}{|a_n|} = q$ (or $\infty$). 

\begin{quote}
\begin{description}
\item[(i)] If $q < 1$, then the series is absolutely convergent.
\item[(ii)] If $q > 1$ (including $q = \infty$ ), then the series is divergent.
\item[(iii)] If $q = 1$, then the series may either converge or diverge; i.e., the test fails.
\end{description}
\end{quote}
\end{theorem}


\begin{proof}
Suppose, first of all, that $\lim_{n \rightarrow \infty} \frac{|a_{n+1}|}{|a_n|} = q < 1$. This implies that the ratio $\frac{|a_{n+1}|}{|a_n|}$ is arbitrarily close to $q$ if $n$ is sufficiently large. Hence if we pick an arbitrary number $r$ such that $q < r < 1$, then there exists an integer $N \geq m$ such that
\begin{equation}
\frac{|a_{n+1}|}{|a_n|} \leq r, \;\;\;\mbox{for every integer}\; n \geq N. 
\label{eq9.5.2}
\end{equation}
We shall show by mathematical induction that (2) implies that
\begin{equation}
|a_{N+i}| \leq r^{i}|a_N|, \;\;\;\mbox{for every integer}\; i \geq 0. \label{eq9.5.3}
\end{equation}
If $i = 0$, then the inequality in (3) becomes $|a_{N+0}| \leq r^{0}|a_N|$, which is true. In the second part of an inductive proof we need to show that, if the inequality
(3) is true for $i = k$, then it is also true for $i = k + 1$. The assumption. then, is that
\begin{equation}
|a_{N+1}| \leq r^k |a_N|, 
\label{eq9.5.4}
\end{equation}
and we want to prove that
$$
|a_{N+k+1}| \leq r^{k+1} |a_N| .
$$
If we multiply both sides of inequality (4) by the positive number $r$, we get
\begin{equation}
r |a_{N+k}| \leq r^{k+1} |a_N| .
\label{eq9.5.5}
\end{equation}
But, inequality (2) tells us that  
$$
\frac{|a_{N+k+1}|}{|a_{N+k}|} \leq r,
$$
and hence that
\begin{equation}
|a_{N+k+1}| \leq r|a_{N+k}|
\label{eq9.5.6}
\end{equation}
Combining inequalities (5) and (6) we have
$$
|a_{N+k+1}| \leq r^{k+1} |a_{N}|,
$$
completing the inductive proof. Since $|r|< 1$, the geometric series $\sum_{i=0}^\infty |a_N| r^i$ converges, and it follows from (3) by the Comparison Test that the series $\sum_{i=0}^\infty |a_{N+i}|$ converges. However,
$$
\sum_{i=0}^\infty |a_{N+i}| = \sum_{i=N}^\infty |a_i| ,
$$
and the convergence of $\sum_{i=N}^{\infty} |a_i|$ implies the convergence of $\sum_{i=m}^\infty |a_i|$. Hence the series $\sum_{i=m}^\infty a_i$ converges absolutely, and the proof of part (i) of the theorem is complete.

We next assume that $\lim_{n \rightarrow \infty} |\frac{a_{n+1}}{a_n} |= q > 1$, and let $r$ be an arbitrary number such that $1 < r < q$. Then there exists an integer $N \geq m$ such that
$$
\frac{|a_{n+1}|}{|a_n|} \geq r, \;\;\;\mbox{for every integer}\; n \geq N.
$$
In the same way in which we proved that (2) implies (3), it follows by induction from the preceding inequality that
$$
|a_{N+i}| \geq r^i |a_N|, \;\;\;\mbox{for every integer}\; i \geq 0.
$$
Since $r > 1$, we know that $\lim_{i \rightarrow \infty} r^i = \infty$ (see Problem 5, page 481), and therefore also that 
$$
\lim_{n \rightarrow \infty} |a_n| = \lim_{n \rightarrow \infty} |a_{N+i} | = \infty .
$$
However, if the series $\sum_{i=m}^\infty a_i$ converges, then it necessarily follows that $\lim_{n \rightarrow \infty} |a_n| = \lim_{n \rightarrow \infty} a_n = 0$. [See Theorem (2.1), page 483, and Problem 2, page 502.] Hence $\sum_{i=m}^\infty a_i$ diverges, and part (ii) is proved.

Part (iii) is proved by giving an example of an absolutely convergent series and one of a divergent series such that $q = 1$ for each of them. Consider the convergent $p$-series $\sum_{i=1}^\infty \frac{1}{i^2}$, which, being nonnegative, is also absolutely convergent. Setting $a_n = \frac{1}{n^2}$, we obtain
$$
a_{n+1} = \frac{1}{(n+ 1)^2} = \frac{1}{n^2 + 2n + 1}
$$
and
$$
\frac{|a_{n+1}|}{|a_n|} = \frac{a_{n+1}}{a_n} = \frac{n^2}{n^2 + 2n + 1} = \frac{1}{1 + \frac{2}{n} + \frac{1}{n^2}}  .
$$
Hence
$$
\lim_{n \rightarrow \infty} \frac{|a_{n+1}|}{|a_n|} = \lim_{n \rightarrow \infty} \frac{1}{1 + \frac{2}{n} + \frac{1}{n^2}} = 1.
$$ 
For the second example, we take the divergent harmonic series $\sum_{i=1}^\infty \frac{1}{i}$. If we let $a_n = \frac{1}{n}$, then $a_{n+1} = \frac{1}{n + 1}$ and
$$
\frac{|a_{n+1}|}{|a_n|} = \frac{a_{n+1}}{a_n} = \frac{n}{n + 1} = \frac{1}{1 + \frac{1}{n}} .
$$
For this series we also get
$$
\lim_{n \rightarrow \infty} \frac{|a_{n+1}|}{|a_n|} = \lim_{n \rightarrow \infty} \frac{1}{1 + \frac{1}{n}} = 1.
$$
The Ratio Test is therefore inconclusive if`$q = 1$, and this completes the proof.
\end{proof}

If $n$ is an arbitrary positive integer, the product $n(n - 1) \cdots 3 \cdot  2 \cdot 1$ is called \textbf{$n$ factorial} and is denoted by $n!$ Thus $3! = 3 \cdot 2 \cdot 1 = 6$ and $5! = 5 \cdot 4 \cdot 3 \cdot 2 \cdot 1 = 120$. Although it may seem strange, $0!$ is also defined and has the value 1. A convenient recursive definition of the factorial is given by the formulas
\begin{eqnarray*}
        0! &=&  1,\\
(n + 1)! &=& (n + 1)n!, \;\;\;\mbox{for every integer}\; n \geq 0.
\end{eqnarray*}

%EXAMPLE 2. 
\begin{example} 
Prove that the following series converges:
$$
\sum_{n=0}^\infty \frac{1}{n!} = 1 + 1 + \frac{1}{2!} + \frac{1}{3!} + \cdots .
$$
%508 INFINITE SERIES [CHAP. 9 
\noindent We write the series as $\sum_{i=0}^{\infty} a_n $ by defining $a_n = \frac{1}{n!}$ for every integer $n \geq 0$. Then 

\begin{eqnarray*}
\frac{|a_{n+1}|}{|a_n|} 
&=& \frac{\frac{1}{(n+1)!} }{\frac{1}{n!}} = \frac{n!}{(n+1)!}\\
&=&\frac{n!}{(n+1)n!} = \frac{1}{n+1} .
\end{eqnarray*}

\noindent Hence
$$
q = \lim_{n \rightarrow \infty} \frac{|a_{n+1}|}{|a_n|}  = \lim_{n \rightarrow \infty} \frac{1}{n + 1} = 0.
$$

\noindent Since $q < 1$, it follows from the Ratio Test that the series is absolutely convergent. But absolute convergence implies convergence [Theorem (5.1)], and we conclude that the series $\sum_{n=0}^{\infty} \frac{1}{n!}$ converges.
\end{example}

%EXAMPLE 3. 
\begin{example} Show that the infinite series 
$$
\sum_{i=1}^{\infty} ir^{i-1}  = 1 + 2r + 3r^2 + 4r^3 + \cdots
$$
\noindent  converges absolutely if $|r| < 1$ and diverges if $|r| \geq 1$. This series is related to the geometric series $\sum_{i=0}^{\infty} r^i = 1 + r + r^2 + \cdots $, and in a later section we shall make use of the relationship. To settle the immediate question of convergence, however, we set $a_i = ir^{i-1}$ for every positive integer $i$, and write the series as $\sum_{i=1}^{\infty} a_i$. Observe, first of all, that if $|r| \geq 1$, then $|a_n| = n|r|^{n-1}$ and
$$
\lim_{n \rightarrow \infty} |a_n| = \lim_{n \rightarrow \infty} n|r|^{n-1} = \infty .
$$
\noindent Hence, if $|r| \geq 1$, the series must diverge, since convergence would imply $\lim_{n \rightarrow \infty} |a_n| = 0$. This proves the second part of what is asked, and we now assume that $|r| < 1$. lf $r = 0$, the series is absolutely convergent with value 1, so we further assume that $r \neq 0$. Then
$$
\frac{|a_{n+1}|}{|a_n|} = \frac{(n+1) |r|^n}{n |r|^{n-1}} = \frac{n+1}{n} |r| = (1 + \frac{1}{n}) |r|,
$$
\noindent and so 
$$
\lim_{n \rightarrow \infty} \frac{|a_{n+1}|}{|a_n|} = \lim_{n \rightarrow \infty}  (1 + \frac{1}{n}) |r| = |r|.
$$
%SEC. 5] ABSOLUTE AND CONDITIONAL CONVERGENCE  509
\noindent Thus $q = |r| < 1$, and the Ratio Test therefore implies that the series is absolutely convergent.
\end{example}

The next theorem, with which we conclude the section, establishes a ttseful inequality.


\begin{theorem} %(5.3) 
If the series $\sum_{i=m}^{\infty} a_i$ converges, then $|\sum_{i=m}^{\infty} a_i| \leq \sum_{i=m}^{\infty} |a_i|$.
\end{theorem}

The result is true even if $\sum_{i=m}^{\infty} a_i$ is not absolutely convergent, for in that case $\sum_{i=m}^{\infty} |a_i| = \infty$, and the inequality becomes $\sum_{i=m}^{\infty} |a_i| \leq \infty$.

\begin{proof}
In view of the preceding remark, we shall assume throughout the proof that $\sum_{i=m}^{\infty} |a_i|$ converges. Let $\{s_n\}$ be the sequence of partial sums corresponding to the series $\sum_{i=m}^{\infty} a_i$. Then
$$
s_n = \sum_{i=m}^{\infty} a_i, \;\;\;\mbox{for every integer}\; n \geq m,  
$$
\noindent and the assumption that $\sum_{i=m}^{\infty} a_i$ converges means that the sequence $\{ s_n \}$ converges and that  
\begin{equation}
\lim_{n \rightarrow \infty} s_n = \sum_{i=m}^{\infty} a_i .  
\label{eq9.5.7}
\end{equation}
The general fact that $|a + b| \leq |a| + |b|$, for any two real numbers $a$ and $b$, can be extended to any finite number of summands, and we therefore have
$$
|s_n| = |\sum_{i=m}^{n} a_i| \leq \sum_{i=m}^{n} |a_i| .
$$
Furthermore,
$$
\sum_{i=m}^{n} |a_i| \leq \sum_{i=m}^{\infty} |a_i|
$$
[see (3.2), page 490, and (1.3)1 page 478). Hence
\begin{equation}
|s_n| \leq \sum_{i=m}^{\infty} |a_i|, \;\;\;\mbox{for every integer}\; n \geq m. 
\label{eq9.5.8}
\end{equation}
It follows from (8) that  
\begin{equation}
|\lim_{n \rightarrow \infty} s_n | \leq \sum_{i=m}^{\infty} |a_i| .
\label{eq9.5.9}
\end{equation}
[It is easy to see that (8) implies (9) if we regard the numbers $s_n$ and $\sum_{i=m}^{\infty} |a_i|$ as points on the line. The geometric statement of (8) is that all the points $s_n$ lie in the closed interval whose endpoints are $-\sum_{i=m}^{\infty} |a_i|$ and $\sum_{i=m}^{\infty} |a_i|$. If
(9) were false, it would mean that $\lim_{n \rightarrow \infty} s_n$ lay outside this interval, a positive distance away from it. But this cannot happen, since Sn is arbitrarily close to $\lim_{n \rightarrow \infty} s_n$ for $n$ sufficiently large.] Combining (7) and (9), we obtain the in-equality which was to be proved.
\end{proof}
